\documentclass[11pt]{amsart}
\usepackage{amscd, amsmath, amssymb, amsthm, amsfonts, amstext, verbatim, enumitem, mathtools, xfrac, nameref, thmtools, hyperref}
\usepackage[utf8]{inputenc}
\usepackage[backend=biber, citestyle=numeric-comp, natbib=true, sortlocale=en_US, url=false, doi=false, eprint=true, maxbibnames=4]{biblatex}            
\usepackage[activate={true,nocompatibility}]{microtype}
\usepackage[capitalize]{cleveref}
\usepackage[english]{babel} 
\usepackage{stmaryrd}
\usepackage[matrix,arrow,curve]{xy}

%Bibliography
\renewbibmacro{in:}{\ifentrytype{article}{}{\printtext{\bibstring{in}\intitlepunct}}}
\newbibmacro{string+doi}[1]{\iffieldundef{doi}{\iffieldundef{url}{#1}{\href{\thefield{url}}{#1}}}{\href{http://dx.doi.org/\thefield{doi}}{#1}}}
\DeclareFieldFormat{title}{\usebibmacro{string+doi}{\mkbibemph{#1}}}
\DeclareFieldFormat[article, inproceedings, inbook, thesis]{title}{\usebibmacro{string+doi}{\mkbibquote{#1}}}
\addbibresource{paper.bib}
            
\newlist{thmlist}{enumerate}{1} \setlist[thmlist]{label=(\roman{thmlisti}), ref=\thethm.(\roman{thmlisti}),noitemsep} \Crefname{thmlisti}{Theorem}{Theorems}
\newlist{lemlist}{enumerate}{1} \setlist[lemlist]{label=(\roman{lemlisti}), ref=\thelemma.(\roman{lemlisti}),noitemsep} \Crefname{lemlisti}{Lemma}{Lemmas}

\theoremstyle{plain} \declaretheorem[name=Theorem, Refname={Theorem,Theorems}]{tm} \Crefname{tm}{Theorem}{Theorems}
\numberwithin{equation}{section}

\newtheorem{lm}{Lemma} \numberwithin{lm}{section} \Crefname{lm}{Lemma}{Lemmas}
\newtheorem{cl}[lm]{Corollary} \Crefname{cl}{Corollary}{Corollaries}
\newtheorem{prop}[lm]{Proposition} \Crefname{prop}{Proposition}{Propositions}
\newtheorem*{tm*}{Theorem}
\newtheorem*{lm*}{Lemma}
\theoremstyle{definition} \newtheorem{df}[lm]{Definition} \Crefname{df}{Definition}{Definitions}
\newtheorem{example}[lm]{Example} \Crefname{example}{Example}{Examples}
\theoremstyle{remark} \newtheorem{rk}[lm]{Remark} \Crefname{rk}{Remark}{Remarks}
\newtheorem*{mtm}{Main Theorem}
\newtheorem*{atm}{Another Presentation}


%Page layout
\textwidth 16cm 
\textheight 22cm 
\headheight 0.5cm 
\evensidemargin 0.3cm 
\oddsidemargin 0.2cm

\newcommand{\Max}{\mathop{\mathrm{Max}}\nolimits}
\newcommand{\proj}{\mathop{\mathrm{proj}}\nolimits}
\newcommand{\Card}{\mathop{\mathrm{Card}}\nolimits}
\newcommand{\Ker}{\mathop{\mathrm{Ker}}\nolimits}
\newcommand{\Cent}{\mathop{\mathrm{Cent}}\nolimits}
\newcommand{\E}{{\mathrm{E}}}
\newcommand{\Um}{\mathop{\mathrm{Um}}\nolimits}
\newcommand{\St}{\mathop{\mathrm{St}}\nolimits}
\newcommand{\Sp}{\mathop{\mathrm{Sp}}\nolimits}
\newcommand{\Ep}{\mathop{\mathrm{Ep}}\nolimits}
\newcommand{\GL}{\mathop{\mathrm{GL}}\nolimits}
\newcommand{\SL}{\mathop{\mathrm{SL}}\nolimits}
\newcommand{\Kt}{\mathop{\mathrm{K_2}}\nolimits}
\newcommand{\Ko}{\mathop{\mathrm{K_1}}\nolimits}
\newcommand{\Ho}{\mathop{\mathrm{H_1}}\nolimits}
\newcommand{\Ht}{\mathop{\mathrm{H_2}}\nolimits}
\newcommand{\epi}{\twoheadrightarrow}
\newcommand{\sgn}{\mathrm{sgn}}
\newcommand{\eps}[1]{\varepsilon_{#1}}
\newcommand{\sign}[1]{\mathrm{sign}(#1)}
\newcommand{\lan}{\langle}
\newcommand{\ran}{\rangle}
\newcommand{\inv}{^{-1}}
\newcommand{\ur}[1]{\!\,^{(#1)}U_1}
\newcommand{\ps}[1]{\!\,^{(#1)}\!P_1}
\newcommand{\ls}[1]{\!\,^{(#1)}\!L_1}
\newcommand{\rA}{\mathsf{A}}
\newcommand{\rB}{\mathsf{B}}
\newcommand{\rC}{\mathsf{C}} 
\newcommand{\rD}{\mathsf{D}} 
\newcommand{\rE}{\mathsf{E}}
\newcommand{\rF}{\mathsf{F}}
\newcommand{\rG}{\mathsf{G}}

\renewcommand{\labelenumi}{\theenumi{\rm)}}
\renewcommand{\theenumi}{\alph{enumi}}

\newcommand\restr[2]{{\left.\kern-\nulldelimiterspace #1 \vphantom{\big|}\right|_{#2}}}
\newcommand*\MapsTo{\xrightarrow[\raisebox{0.25 em}{\smash{\ensuremath{\sim}}}]{}}  

\title{On centrality of orthogonal $\Kt$}
\keywords {Steinberg groups, $K_2$-functor. {\em Mathematical Subject Classification (2010):} 19C09}

\author{Sergey Sinchuk}
\email{sinchukss {\it at} gmail.com}

\author {Andrei Lavrenov}
\email {avlavrenov {\it at} gmail.com}

\date {\today}
\thanks{Authors of the present paper acknowledge the financial support from Russian Science Foundation grant 14-11-00297}

\begin{document}

\begin{abstract} We give a short uniform proof of centrality of $\mathrm K_2(\Phi, R)$ for all simply-laced root systems $\Phi$ of rank $\geq 3$.
\end{abstract}

\maketitle

%\begin{thebibliography}{99}
%\bibitem{Lav2} A.~Lavrenov, A local--global principle for symplectic $\mathrm K_2$, {\it to appear}.
%\bibitem{Rao} *Statement of Ravi Rao's hypothesis
%\bibitem{Swa1} *Swan's relative Steinberg group 1
%\bibitem{Swa2} *Swan's relative Steinberg group 2
%\bibitem{Swa3} *Swan's relative Steinberg group 3
%\bibitem{Tul0} *Tulenbaev's paper on almost commutative rings (cite somewhere)
%\end{thebibliography}

\section*{Inroduction}
The main goal of the present paper is to obtain a uniform proof of the local-global principle for $\Kt$ and centrality of $\Kt$ for Chevalley groups of type $ADE$.

Recall that for each reduced root system $\Phi$ and a commutative ring $R$ one can construct a split simple group $G(\Phi, R)$ called \emph{Chevalley group}.
For $\Phi$ of rank $\geq 2$ the abstract group $G(\Phi, R)$ contains a large normal subgroup $\E(\Phi, R)$ generated by the elementary root unipotents $t_\alpha(\xi)$, $\xi\in R$, $\alpha\in \Phi$.
The quotient group $\Ko(\Phi, R)=G(\Phi, R)/\E(\Phi, R)$ is called the \emph{unstable $\Ko$-functor}.
This functor shares many properties with its stable counterpart $\mathrm{SK}_1(R) = \lim\limits_{l\to\infty}\Ko(\rA_\ell, R)$ which, in turn, is an essential direct summand of $\Ko(R)$.

Similarly to the definition of the algebraic $\Kt$-functor, one can define $\Kt(\Phi, R)$ as the kernel of the canonical projection $\varphi\colon\St(\Phi, R)\twoheadrightarrow \E(\Phi, R)$.
Here $\St(\Phi, R)$ stands for the \emph{Steinberg group} of type $\Phi$, i.\,e. the group defined by formal generators $x_\alpha(\xi)$ modeling root unipotents $t_\alpha(\xi)$
and the explicit list of relations called Steinberg relations. The projection $\varphi$ is defined by the identity $\varphi(x_\alpha(\xi))=t_\alpha(\xi)$.

It is classically known that $\Kt(\Phi, R)$ is a central subgroup of $\St(\Phi, R)$ if $R$ is commutative local ring and $\Phi$ has rank $\geq 2$ (see \cite{Stein}).
However, to prove that centrality of $\Kt$ holds for \emph{arbitrary} $R$ is much harder.
The first result in this direction was obtained in 1977 by W. van~der~Kallen for $\Phi=\rA_\ell$, $\ell\geq 3$.
Using a similar method the first-named author established centrality of $\Kt(\rC_\ell, R)$, see \cite{Lav}.
On the other hand, the second-named author has obtained centrality of $\Kt(\rE_\ell, R)$ using a different method, see \cite{SCh}.
In turn, our main result is the following theorem.
\begin{tm}\label{tm:main} Let $R$ be an arbitrary commutative ring.
 If $\Phi$ is an irreducible root system of type $\Phi=\rA_\ell, \rD_\ell, \rE_\ell$, $\ell\geq 3$ then the following statements are true:
 \begin{enumerate}
  \item The group $\St(\Phi, R)$ satisfies local-global principle for $\Kt$, i.\,e. an element $g\in\St(n,\,R[X])$ such that $g(0)=1\in\St(n,\,R)$
  is trivial if and only if $g_M=1\in\St(n,\,R_M[X])$ for all maximal ideals $M\triangleleft R$.
  \item The group $\Kt(\Phi, R)$ is contained in the centre of $\St(\Phi, R)$.  
 \end{enumerate}
\end{tm}
The special case $\Phi=\rA_\ell$, $\ell\geq 3$ of the first statement of the above theorem was proven for the first time by Tulenbaev~\cite[Th.~2.1]{Tul}, while the case $\Phi=\rE_\ell$ was studied in~\cite{SCh}.
Thus, the new cases in \cref{tm:main} are $\Phi=\rA_3$ and $\Phi=\rD_\ell$, $\ell\geq 4$.

The paper is organised as follows. In the first section we give yet another presentation for $\St(4,\,R,\,I)=\St(\mathrm A_3,\,R,\,I)$.
It mimics the standart vector representation of $\SL(n,\,R)$, therefore in the first section we use representations and corresponding vector notation. 
Afterwards we can deduce a local--global principle for $\St(\mathrm\Phi,\,R)$ and centrality of $\Kt(\mathrm\Phi,\,R)$ for $\Phi=\mathrm A_3$. 
In the second section we use obtained results to prove the same theorems for any simply laced $\Phi$ of rank $\geq3$. There we switch to the root notation.

\section{Principal notation}

Tulenbaev used {\it another presentation} for a relative Steinberg group to prove a local--global principle for a linear Steinberg group~\cite{Tul}. 
This relative another presentation was inspired by the {\it absolute} one of van der Kallen~\cite{vdK}. 
However, Tulenbaev's presentation can only be used for $\St(l+1,\,R)=\St(\mathrm A_l,\,R)$, $l\geq4$.

Tulenbaev's result can be applied to the subsystems of type $\mathrm A_4$ inside $\mathrm E_l$ and ``glued'' to get a local--global principle for Steinberg groups of type $\mathrm E_l$~\cite{SCh}. 
To use this glueing method for groups of type $\mathrm D_l$ as well one has to work with subsystems of type $\mathrm A_3$ instead, and thus extend Tulenbaev's result to the case of $\mathrm A_3$.

Recently, the local--global principle was obtained for Steinberg groups of type $\mathrm C_l$ starting from rank 3~\cite{Lav2}. 
That is another motivation to consider the linear Steinberg group in rank 3.

In the present paper $R$ denotes commutative (associative, unital) ring, $l\geq3$, and $n=l+1\geq4$, $R^n$ is a free (right) $R$-module with basis $e_1,\ldots,e_n$, 
$\GL(n,\,R)$ is a group of invertible $n\times n$ matrices over $R$, $\E(n,\,R)$ its elementary subgroup generated by elementary transvections 
$$\{t_{ij}(r)=1+e_{ij}r\mid1\leq i\neq j\leq n,\ r\in R\},$$
where 1 stands for the identity matrix and $e_{ij}$ for the matrix unit. The Steinberg group is a group given by formal generators
$$\{x_{ij}(r)\mid1\leq i\neq j\leq n,\ r\in R\}$$
and Steinberg relations
\setcounter{equation}{0}
\renewcommand{\theequation}{S\arabic{equation}}
\begin{align}
&x_{ij}(r)x_{ij}(s)=x_{ij}(r+s),\\
&[x_{ij}(r),\,x_{hk}(s)]=1,\text{ for }h\neq j,\ k\neq i,\\
&[x_{ij}(r),\,x_{jk}(s)]=x_{ik}(rs).
\end{align}
All commutators in the present paper are left-normed, $[x,\,y]=xyx\inv y\inv$. 
Denote $\phi\colon\St(n,\,R)\rightarrow\E(n,\,R)$ a natural projection sending $x_{ij}(r)$ to $t_{ij}(r)$.

Suslin showed that (already for $n\geq3$) the elementary group coincides with the subgroup of $\GL(n,\,R)$ generated by matrices
$$\{t(u,\,v)=1+uv^t\mid u\in\Um(n,\,R),\ v\in R^n,\ u^tv=0\},$$
where $\Um(n,\,R)$ is a set of unimodular coloumns of $R^n$, i.e., such that their entries generate the unit ideal of $R$, $u^t$ stands for $u$ {\it transposed}~\cite{Sus}.
This result clearly implies that $\E(n,\,R)$ is normal inside $\GL(n,\,R)$.
Then, van der Kallen developed Suslin's ideas and showed that the Steinberg group $\St(n,\,R)$ is isomorphic to a group defined by a set of generators
$$\{X(u,\,v)\mid u\in\Um(n,\,R),\ v\in R^n,\ u^tv=0\}$$ and relations
\setcounter{equation}{0} \renewcommand{\theequation}{K\arabic{equation}}
\begin{align}
&X(u,\,v)X(u,\,w)=X(u,\,v+w),\\
&X(u,\,v)X(u',\,v')X(u,\,v)\inv=X(t(u,\,v)u',\,t(v,\,u)\inv v')
\end{align}
for $n\geq4$~\cite{vdK}.
Then, it is also obvious that $\phi\colon\St(n,\,R)\rightarrow\E(n,\,R)$ is a central extension.
Observe that van der Kallen uses pairs of a coloumn and a row $(i,\,j)$ to parametrise his generators, and we prefer two coloumns.

The local--global principle for the elementary group proven by Suslin~\cite{Sus} is a statement that an elementary matrix over a polynomial ring $g\in\E(n,\,R[X])$ becomes trivial after the evaluation ``$X=0$''
$$
g\in\Ker\big(\E(n,\,R[X])\rightarrow\E(n,\,R)\big)
$$
and after localisations
$$
g_{\mathfrak m}=1\in\E(n,\,R_{\mathfrak m}[X])
$$
for all maximal ideals $\mathfrak m$ of a base ring $R$, if and only if it is trivial from the very begining
$$
g=1\in\E(n,\,R[X]).
$$

For $I\trianglelefteq R$ denote $\St(n,\,R,\,I)=\Ker\big(\St(n,\,R)\epi\St(n,\,R/I)\big)$ {\it a relative Steinberg group}. Clearly, $\St(n,\,R[X],\,XR[X])$ is the kernel of the evaluation ``$X=0$''.

\section{Yet another presentation for $\St(4, R)$.}

The key step in Tulenbaev's proof is constructing a homomorphism
$$
\mathrm T\colon\St(n,\,R_a[X],\,XR_a[X])\rightarrow\St(n,\,R\ltimes XR_a[X],\,XR_a[X])
$$
for any (non-nilpotent) element $a\in R$, $n\geq5$ ($R_a$ stands for a principle localisation of $R$ in $a$). This $\mathrm T$ is an inverse to a map induced by localisation, so that constructing $\mathrm T$ may be treated as a proof of excision property for Steinberg group in this very particular situation.

To prove the local--global principle for Steinberg groups of type $\mathrm D_l$ and centrality of $\mathrm K_2(\mathrm D_l,\,R)$ we need such a map for $n=4$ (i.e., for the group of type $\mathrm A_3$).

Tulenbaev uses the following presentaion for the relative Steinberg group, see~\cite[Prop.~1.6]{Tul}. Assume that $I\trianglelefteq R$ such that a short exact sequence
\renewcommand{\theequation}{$\dagger$}
\begin{align}
\xymatrix{
0\ar@<0.1ex>[r]^{}&I\ar@<0.1ex>[r]^{}&R\ar@<0.5ex>[r]^{\pi}&R/I\ar@<0.5ex>[l]^{\sigma}\ar@<0.1ex>[r]^{}&0
}
\end{align}
splits, i.e., $\pi\sigma=1$. The $\St(n,\,R,\,I)$ is isomorphic to a group given by generators
$$
\{X(u,\,v)\mid u\in\E(n,\,R)e_1,\ v\in I^n,\ u^tv=0\}
$$
subject to relations
\setcounter{equation}{0}
\renewcommand{\theequation}{T\arabic{equation}}
\begin{align}
&X(u,\,v)X(u,\,w)=X(u,\,v+w),\\
&X(u,\,v)X(u',\,v')X(u,\,v)\inv=X(t(u,\,v)u',\,t(v,\,u)\inv v'),\\
&X(ur+w,\,v)=X(u,\,vr)X(w,\,v)\,\text{ for }r\in R,\ (u,\,w)\in\Um_{n\times2}(R).
\end{align}
Here $\Um_{n\times2}(R)$ is a set of $n\times2$ matrices $M$ such that there exists a $2\times n$ matrix $N$ with $NM=\begin{pmatrix}1&0\\0&1\end{pmatrix}$. The ideal satisfying $\dagger$ is called the splitting ideal, $XR_a[X]\trianglelefteq R_a[X]$ is obviously the splitting one.

Another presentations of van der Kallen and Tulenbaev are given in terms of generators parametrised by pairs of vectors, where the first one is ``good'' (unimodular or a coloumn of an elementary matrix) and the second one is arbitrary. This allows to formulate easily an additivity by the second argument, but one has to write something similar to T3 instead of additivity by the first one.

Now, we need to construct a homomorphism
$$
\mathrm T\colon\St(n,\,R_a[X],\,XR_a[X])\rightarrow\St(n,\,R\ltimes XR_a[X],\,XR_a[X]),
$$
i.e., to find elements in the target group subject to Tulenbaev's relations, but because of T3 one can do it only for $n\geq5$, see~\cite[Lemmas~1.2 and~1.3\,c)]{Tul}.

One can not parametrise generators by a pair of arbitrary vectors since it is not even known that $t(u,\,v)\in\E(n,\,R)$ for arbitrary orthogonal $u$ and $v$~\cite{Rao}. 

To generalise Tulenbaev's results for $n=4$ we use a presentation with two types of generators: $F(u,\,v)$ with $u$ good and $v$ arbitrary, which are additive by the second component, and $S(u,\,v)$ with $u$ arbitrary and $v$ good, which are additive by the first component. For $u$ and $v$ both good we require that these two types of generators coincide. More precisely,

\begin{df}
for $I\trianglelefteq R$, $n\geq4$ define $\St^*(n,\,R,\,I)$ to be a group given by the set of generators
\begin{multline*}
\{F(u,\,v)\mid u\in\E(n,\,R)e_1,\ v\in I^n,\ u^tv=0\}\,\cup\\
\cup\{S(u,\,v)\mid u\in I^n,\ v\in\E(n,\,R)e_1,\ u^tv=0\}
\end{multline*}
subject to relations
\setcounter{equation}{0}
\renewcommand{\theequation}{R\arabic{equation}}
\begin{align}
&F(u,\,v)F(u,\,w)=F(u,\,v+w),\\
&S(u,\,v)S(w,\,v)=S(u+w,\,v),\\
&F(u,\,v)F(u',\,v')F(u,\,v)\inv=F(t(u,\,v)u',\,t(v,\,u)\inv v'),\\
&F(u,\,va)=S(ua,\,v)\ \forall\,a\in I,\,(u,\,v^t)=(Me_1,\,e_2^tM\inv),\,M\!\in\E(n,\,R).
\end{align}
\end{df}

One could probably wonder why we assumed only one relation on conjugated element, instead of four different relations on generators of each type conjugated by generators of each type. Three ``missing'' relations actualy hold automaticaly.

\begin{lm}
\label{allyouneedisf}
Denote $\varphi\colon\St^*(n,\,R,\,I)\rightarrow\E(n,\,R)$ the natural map, sending $F(u,\,v)\mapsto t(u,\,v)$, $S(u,\,v)\mapsto t(u,\,v)$.
\begin{enumerate}
\item 
$\St^*(n,\,R,\,I)$ is generated by the set of elements
$$
\{F(u,\,va)\mid a\in I,\ (u,\,v^t)=(Me_1,\,e_2^tM\inv),\,M\in\E(n,\,R)\};
$$
\item
for any $g\in\St^*(n,\,R,\,I)$ holds
\setcounter{equation}{2}
\renewcommand{\theequation}{R\arabic{equation}'}
\begin{align}
&gF(u,\,v)g\inv=F(\varphi(g)u,\,\varphi(g\inv)^tv);
\end{align}
\item
for any $g\in\St^*(n,\,R,\,I)$ holds
\setcounter{equation}{2}
\renewcommand{\theequation}{R\arabic{equation}''}
\begin{align}
&gS(u,\,v)g\inv=S(\varphi(g)u,\,\varphi(g\inv)^tv);
\end{align}
\item
there exists a well-defined ``transpose automorphism''
$$
F(u,\,v)^t=S(v,\,u),\quad S(u,\,v)^t=F(v,\,u).
$$
\end{enumerate}
\end{lm}

\begin{proof}
Take an arbitrary generator of the first type $F(u,\,v)$ and take $M\in\E(n,\,R)$ such that $F(u,\,v)=F(Me_1,\,M^*\tilde v)$. Here $M^*$ stands for a contragradient matrix, $M^*=(M\inv)^t$. By R1 we have
$$
F(u,\,v)=\prod\limits_{k\neq1}F(Me_1,\,M^*e_k\tilde v_k)
$$
where $\tilde v_k$ stands for $k$-th coordinate of $\tilde v=\sum e_k\tilde v_k$. Similarly, with the use of R4 we get 
$$
S(u,\,v)=\prod\limits_{k\neq1}S(Ne_k\tilde u_k,\,N^*e_1)=\prod\limits_{k\neq1}F(Ne_k,\,N^*e_1\tilde u_k)
$$
and obtain a). Obviously, b) follows from a). For c) it is sufficient to show that
$$
F(u,\,v)S(u',\,v')F(u,\,v)\inv=S(t(u,\,v)u',\,t(u,\,v)^*v').
$$
Let $S(u',\,v')=\prod F(Ne_k,\,N^*e_1\tilde u_k)$, then
\begin{multline*}
F(u,\,v)S(u',\,v')F(u,\,v)\inv=F(u,\,v)\prod F(Ne_k,\,N^*e_1\tilde u_k)F(u,\,v)\inv=\\
=\prod F(t(u,\,v)Ne_k,\,t(u,\,v)^*N^*e_1\tilde u_k)=\\
=\prod S(t(u,\,v)Ne_k\tilde u_k,\,t(u,\,v)^*N^*e_1)=S(t(u,\,v)u',\,t(u,\,v)^*v').
\end{multline*}
Finally, d) follows from c).
\end{proof}

Next, we show that for a splitting ideal $I\trianglelefteq R$ the group $\St^*(n,\,R,\,I)$ is isomorphic to $\St(n,\,R,\,I)=\Ker\big(\St(n,\,R)\epi\St(n,\,R/I)\big)$. With this end we construct two inverse homomorphisms
$$
\xymatrix{
\St^*(n,\,R,\,I)\ar@<0.5ex>[r]^{\iota}&\St(n,\,R,\,I)\ar@<0.5ex>[l]^{\kappa}.
}
$$

Firstly, we define $\iota$ as a homomorhism to an absolute group
$$
\xymatrix{
\St^*(n,\,R,\,I)\ar@<0.0ex>[r]^{\iota}&\St(n,\,R).
}
$$
We do not need an assumption that $I$ is a splitting ideal for that.

Clearly, we map elements $F(u,\,v)$ to van der Kallen's another generators $X(u,\,v)$. To find images for $S(u,\,v)$ as well we recall that van der Kallen in~\cite[3.8--3.10]{vdK} defines also elements
$$
x(u,\,v)\in\St(n,\,R),\ u^tv=0,\ u_i=0\,\text{ or }\,v_i=0\,\text{ for some }\,1\leq i\leq n.
$$
Now, ``transpose'' the definition~\cite[3.13]{vdK}.

\begin{df}
For $u\in R^n$, $v\in\E(n,\,R)e_1$, $u^tv=0$, consider the set $\overline Y(u,\,v)\subseteq\St(n,\,R)$ consisiting of all elements $y\in\St(n,\,R)$ which could be decomposed as a product $\prod x(w^k,\,v)$, where $\sum w^k=u$ and $w^k=(e_pv_q-e_qv_p)c^k$ for some $c^k\in R$, $1\leq p\neq q\leq n$.
\end{df}

Since coloumns of elementary matrices are unimodular, $\overline Y(u,\,v)$ is not empty (see~\cite[3.1--3.2]{vdK}). Obviously, for $x\in\overline Y(u,\,v)$ and $y\in\overline Y(w,\,v)$ holds $xy\in\overline Y(u+w,\,v)$.

Repeating~\cite[3.14--3.15]{vdK} verbatim one shows that

\begin{lm}
for $g\in\St(n,\,R)$ holds
\begin{enumerate}
\item
$g\overline Y(u,\,v)g\inv\subseteq\overline Y(\phi(g)u,\,\phi(g)^*v)$;
\item
$\overline Y(u,\,v)$ consists of exactly one element.
\end{enumerate}
\end{lm}

The only element of $\overline Y(u,\,v)$ we denote by $Y(u,\,v)$. We map elements $S(u,\,v)\in\St^*(n,\,R,\,I)$ to $Y(u,\,v)$. To show that such a homomorphism is well-defined we check that $X(u,\,v)$ and $Y(u,\,v)$ satisfy relations R1--R4, but R1--R3 are already checked.

\begin{lm}
Consider $(u,\,v^t)=(Me_1,\,e_2^tM\inv)$, $M\in\E(n,\,R)$, $a\in I$. Then $X(u,\,va)=Y(ua,\,v)$.
\end{lm}

\begin{proof}
It suffices to show that $X(e_1,\,e_2a)=Y(e_1a,\,e_2)$. Compute the commutator $[Y(-e_3,\,e_2),\,X(e_1,\,e_3a)]$ in two ways:
$$
Y(-e_3,\,e_2)X(e_1,\,e_3a)Y(-e_3,\,e_2)\inv\cdot X(e_1,\,-e_3a)=X(e_1,\,e_2a),
$$
and
$$
Y(-e_3,\,e_2)\cdot\,X(e_1,\,e_3a)Y(e_3,\,e_2)X(e_1,\,e_3a)\inv=Y(e_1a,\,e_2).
$$
\end{proof}

Therefore, the homomorphism $\iota\colon\St^*(n,\,R,\,I)\rightarrow\St(n,\,R)$ sending $F(u,\,v)$ to $X(u,\,v)$ and $S(u,\,v)$ to $Y(u,\,v)$ is well-defined. Denote $\rho_I\colon R\epi R/I$, then one can easily see that induced morphism on Steinberg groups 
$$\rho_I^*\colon\St(n,\,R)\rightarrow\St(n,\,R/I)$$
 sends $X(u,\,v)$ to $X(\rho_I(u),\,\rho_I(v))$ and $Y(u,\,v)$ to $Y(\rho_I(u),\,\rho_I(v))$. Observe that $\E(n,\,R)\epi\E(n,\,R/I)$ is surjective so that for $x\in\E(n,\,R)e_1$ holds that $\rho_I(x)\in\E(n,\,R/I)e_1$. It follows from additivity relations that $X(u,\,0)=1=Y(0,\,v)$, thus $\mathrm{Im}(\iota)\subseteq\Ker(\rho_I^*)=\St(n,\,R,\,I)$.

We constructed $\iota$ for an arbitrary $I\trianglelefteq R$. For $I$ a splitting ideal we construct an inverse homomorphism
$$
\kappa\colon\St(n,\,R,\,I)\rightarrow\St^*(n,\,R,\,I).
$$
With this end we recall a presentation of $\St(n,\,R,\,I)$ as a group with action of $\St(n,\,R)$. For a splitting ideal such a presentation was firstly obtained by Swan~\cite{Swa1,Swa2,Swa3}, and for an arbitrary $I$ by Keune~\cite{Keu} and Loday~\cite{Lod}.

For a group $G$ acting on a group $H$ from the left denote by $\!\,^gh$ the image of $h\in H$ under homomorphism corresponding by $g\in G$.

Let $I\trianglelefteq R$ be a splitting ideal then $\St(n,\,R,\,I)$ can be presented as a group with action of $\St(n,\,R)$ with the set of (relative) generators
$$
\{y_{ij}(a)\mid a\in I,\ 1\leq i\neq j\leq n\}
$$
subject to relations
\setcounter{equation}{0}
\renewcommand{\theequation}{KL\arabic{equation}}
\begin{align}
&y_{ij}(a)y_{ij}(b)=y_{ij}(a+b),\\
&\!\,^{x_{ij}(r)}y_{hk}(b)=y_{hk}(b),\text{ for }h\neq j,\ k\neq i,\\
&\!\,^{x_{ij}(r)}y_{jk}(b)\cdot y_{jk}(-b)=y_{ik}(rb),\\
&y_{ij}(a)\cdot\,^{x_{jk}(s)}y_{ij}(-a)=y_{ik}(as),\\
&\!\,^{x_{hk}(a)}\big(\!\,^gy_{ij}(b)\big)=y_{hk}(a)\cdot\,^gy_{ij}(b)\cdot y_{hk}(-a)\,\text{ for }a\in I,\ g\in\St(n,\,R).
\end{align}

In other words, $\St(n,\,R,\,I)$ is isomorphic to a quotient group of a free group $F$ on the set of generators 
$$\St(n,\,R)\times\{y_{ij}(a)\mid a\in I,\ 1\leq i\neq j\leq n\}$$
(the natural action of $\St(n,\,R)$ on $F$ is given by $\!\,^f(g,\,y)=(fg,\,y)$) modulo the normal equivariant subgroup generated by KL1--KL5 (where $y_{ij}(a)$ stands for $(1,\,y_{ij}(a))$. Keune in Loday consider only stable Steinberg group $\St(R)=\varinjlim\St(n,\,R)$. This presentation for unstable group firstly appeared in~\cite{SCh} in the context of Chevalley groups. Observe that the definition of the relative Steinberg group in~\cite[Def.~3.3]{SCh} differs from ours but remark after this definition and~\cite[Lem.~8]{SCh} immediantly imply that for a splitting ideal $I\trianglelefteq R$ it coincides with the one given in the present paper. The existance of the presentation is proven in~\cite[Prop.~6]{SCh}. Relation~5 in~\cite[Prop.~6]{SCh} joins relations KL3 and KL4, and relations~2 and 3 follow from KL4 and KL5 with the use of KL6.

One could probably wonder why Tulenbaev needs his own presentation for $\St(n,\,R,\,I)$ and does not use the one of Keune--Loday. The reason is the following. To construct a map from Keune--Loday group one should firstly define an action of the absolute Steinberg group on the target group, in our context, the action of $\St(n,\,B_a)$ on $\St(n,\,B,\,I)$, what is probably possible, but seems to be much harder, then giving another presentation (in particular, one should define $\!\,^{x_{ij}(r/a^m)}y_{ji}(c)$).

Now, to construct
$
\kappa\colon\St(n,\,R,\,I)\rightarrow\St^*(n,\,R,\,I)
$
we need to define an action of $\St(n,\,R)$ on $\St^*(n,\,R,\,I)$ and find elements $y^*_{ij}(a)\in\St^*(n,\,R,\,I)$ subject to KL1--KL5. To define a action of the absolute group we use van der Kallen's another presentaion for it. For $u\in\Um(n,\,R)$, $v\in R^n$, $u^tv=0$ define
$
\alpha(u,\,v)\colon\St^*(n,\,R,\,I)\rightarrow\St^*(n,\,R,\,I)
$
by $\alpha(u,\,v)\big(F(u',\,v')\big)=F(t(u,\,v)u',\,t(u,\,v)^*v')$, and $\alpha(u,\,v)\big(S(u',\,v')\big)=S(t(u,\,v)u',\,t(u,\,v)^*v')$. Obviously, the images of the generators satisfy R1--R4, so that $\alpha(u,\,v)$ is a well-defined automorphism. Also, $\alpha$'s themselves clearly satisfy K1--K2, thus $X(u,\,v)\mapsto\alpha(u,\,v)$ is a well-defined action of $\St(n,\,R)$ on $\St^*(n,\,R,\,I)$.

Next, define $y_{ij}^*(a)=F(e_i,\,e_ja)$, $a\in I$. These elements obviously satisfy KL1, KL2 and KL5. Check KL3:
\begin{multline*}
\!\,^{x_{ij}(r)}y^*_{jk}(b)=F(t_{ij}(r)e_j,\,t_{ji}(-r)e_kb)=F(e_ir+e_j,\,e_kb)=\\
=S(e_irb+e_jb,\,e_k)=S(e_irb,\,e_k)S(e_jb,\,e_k)=F(e_i,\,e_krb)F(e_j,\,e_kb).
\end{multline*} 
KL4 is similar. Finally, we have a well-defined map
$$
\kappa\colon\St(n,\,R,\,I)\rightarrow\St^*(n,\,R,\,I)
$$
sending $y_{ij}(a)$ to $y^*_{ij}(a)$. Obviously, $\iota\circ\kappa=\mathrm{id}$ (cf.~\cite[3.6\,d)]{vdK}), thus $\kappa$ is injective. Surjectivity of $\kappa$ follows from Lemma~\ref{allyouneedisf}\,a), thus it is an isomorphism and $\iota$ is inverse to it.

At this point we obtained the presentation $\St^*(n,\,R,\,I)$ for the $\St(n,\,R,\,I)$ where $I\trianglelefteq R$ is a splitting ideal, $n\geq4$.

Below we obtain the main result of this section, namely, construct a map 
$$
\mathrm T\colon\St(n,\,R_a[X],\,XR_a[X])\rightarrow\St(n,\,R\ltimes XR_a[X],\,XR_a[X]),
$$
for $n\geq4$. A local-global principle for Steinberg group and centrality of $\mathrm K_2$ formally follow from existance of this map~\cite[Lem.~15, Lem.~16, proof of Th.~2]{SCh}.

We work in a more general situation.

\begin{tm}
\label{a3map}
Let $B$ be a ring, $I\trianglelefteq B$, $a\in B$ such that $\forall\,x\in I$ there exists a unique $y\in I$ such that $ya=x$ {\rm(}we denote this $y$ by $\frac xa$, elements $\frac x{a^m}$ are also well defined{\rm)}. This requirement is equivalent to say that the principle localisation $\lambda_a\colon I\rightarrow I_a$ of ideal $I$ is an isomorphism. Then there exists a map
$$
\mathrm T\colon\St(n,\,B_a,\,I)\rightarrow\St(n,\,B)
$$
making the diagram
$$
\xymatrix{
\St^*(2n,\,B,\,I)\ar@<-0.0ex>[rr]^{\iota}\ar@<-0.0ex>[d]_{\lambda_{a}^*}&&\St(2n,\,B)\ar@<-0.0ex>[d]^{\lambda_{a}^*}\\
\St^*(2n,\,B_a,\,I)\ar@<-0.0ex>[rr]_{\iota}\ar@{-->}[rru]_{\mathrm T}&&\St(2n,\,B_a)
}
$$
commutative.
\end{tm}

To prove this theorem we need to find some elements ``$X(u,\,v)$'' inside $\St(n,\,B)$ for $u\in\E(n,\,B_a)e_1$, $v\in I$. The idea is the following. Say $a$ is not a zero divisor, then $ua^m\in B$ for some $m$ and we will construct some element ``$X(ua^m,\,va^{-m})$''. One can not garantee that $ua^m\in\E(n,\,B)e_1a^k$, however, the ideal generated by entries of $ua^m$ contains some power of $a$ (this is equivalent to say that $ua^m$ becomes unimodular after the principle localisation in $a$). Denote $\mathrm I(u)$ the ideal generated by entries of $u\in R^n$, $\mathrm I(u)=\sum\limits_{k=1}^nu_kR$. Then, we need generators parametrised by pairs $(u,\,v)$ with $u^tv=0$, $v\in I$, $a^m\in\mathrm I(u)$ for some $m\in\mathbb N$. Such generators are defined in~\cite{Tul}.

\begin{df}[Tulenbaev]
For $u$, $v\in B^n$, $a\in\mathrm I(u)$ and $v_1,\ldots,v_N\in B^n$ such that $u^tv_k=0$ $\forall\,k$,  each $v_k$ has at least two zero coordinates and $\sum_{k=1}^Nv_k=v$ define $X_{u,v}(a)=\prod\limits_{k=1}^Nx(u,\,v_ka)$. Tulenbaev shows that factors $x(u,\,v_ka)$ commute~\cite[Lem.~1.1\,e)]{Tul} and that for any other decomposition of $v$ as a sum $v=\sum_{j=1}^Mv'_j$ of vectors orthogonal to $u$ and having two zero coordinates holds $\prod_{k=1}^Nx(u,\,v_ka)=\prod_{j=1}^Mx(u,\,v'_ja)$~\cite[p.~3]{Tul}, i.e., elements $X_{u,v}(a)$ are well-defined. Obviously, $\phi(X_{u,v}(a))=t(u,\,va)$.
\end{df}

Observe that Tulenbaev uses different notation for van der Kallen elements. He writes $X_{u,v}$ instead of $X(u,\,v)$ and $X(u,\,v)$ instead of $x(u,\,v)$. We keep van der Kallen's notation.

\begin{lm}
\label{xproperties}
For $u$, $v$, $v'$ and $w\in B^n$, such that $v$ and $v'$ have decomposition as in above definition, $u^tw=0$, $a$, $b\in\mathrm I(u)$, $c\in B$, $g\in\St(n,\,B)$ holds
\begin{enumerate}
\item
$X_{u,vc}(a)=X_{u,v}(ca)$,
\item
$X_{uc,v}(ca)=X_{u,vc^2}(a)$,
\item
$X_{u,v}(a)X_{u,v'}(a)=X_{u,v+v'}(a)$,
\item
$g\,X_{u,wb}(a)g\inv=X_{\phi(g)u,\phi(g)^*wb}(a)$.
\end{enumerate}
\end{lm}

\begin{proof}
The statement of a) is obvious from the definition, b) follows from~\cite[Lem.~1.1\,d)]{Tul}, c) is~\cite[Lem.~1.3\,a)]{Tul}, d) is proven for $n\geq5$ in~\cite[Lem.~1.3\,b)]{Tul} and afterwards Tulenbaev makes a remark that for $n=4$ the statement is also true. Indeed, take $z\in B^n$ such that $z^tu=b$ and decompose
$$
(z^tu)w=\sum_{i<j}w_{ij},
$$
where $w_{ij}=(e_iu_j-e_ju_i)(w_iz_j-w_jz_i)=w_{ji}$. Each $w_{ij}$ is orthogonal to $u$ and has two zero coordinates ($n\geq4$). Thus, $X_{u,wb}(a)=\prod_{i<j}x(u,\,w_{ij}a)$. It is enough to prove the statement of d) for $g=x_{hk}(r)$. If $h\neq i,j$ or $\{h,k\}=\{i,j\}$, $\phi(g)^*w_{ij}a$ still has two zero coordinates. Consider the case $j=h$, $i\neq k$. With~\cite[3.12]{vdK} we get $g\,x(u,\,w_{ij}a)g\inv=x(\phi(g)u,\,\phi(g)^*w_{ij}a)$. Denote $u_{ij}=e_iu_j-e_ju_i$ and $c_{ij}=w_iz_j-w_jz_i$. Using
$$
\phi(g)^*u_{ij}=(\phi(g)u)_{ij}+(\phi(g)u)_{ki}\,r
$$
and~\cite[3.11]{vdK} one obtains
$$
x(\phi(g)u,\,\phi(g)^*w_{ij}a)=x(\phi(g)u,\,(\phi(g)u)_{ij}c_{ij}a)\cdot x(\phi(g)u,\,(\phi(g)u)_{ki}\,rc_{ij}a).
$$
Decomposing in such a way each $\phi(g)^*w_{ij}a$ which does not have two zero coordinates we finally get a product from the definition of $X_{\phi(g)u,\phi(g)^*wb}(a)$.
\end{proof}

Siilarly, one can define ``transposed'' version of Tulenbaev's elements $X_{u,v}(a)$.

\begin{df}
For $u$, $v\in B^n$, $a\in\mathrm I(v)$, and $u_1,\ldots,u_N\in B^n$ such that $u_k^tv=0$, each $u_k$ has at least two zero coordinates and $u=\sum_{k=1}^Nu_k$ define $Y_{u,v}(a)=\prod\limits_{k=1}^Nx(u_ka,\,v)$. Repeating Tulenbaev's argumentation~\cite[p.~3]{Tul} one can show that the definition does not depend on the order of factors and on choice of decomposition of $u$.
\end{df}

Now, one can repeat van der Kallen's and Tulenbaev's argumentation to prove transposed version of Lemma~\ref{xproperties}. We leave this to the reader.

\begin{lm}
\label{yproperties}
For $u$, $u'$, $w$ and $v\in B^n$, such that $u$ and $u'$ have decomposition as in above definition, $w^tv=0$, $a$, $b\in\mathrm I(v)$, $c\in B$, $g\in\St(n,\,B)$ holds
\begin{enumerate}
\item
$Y_{uc,v}(a)=Y_{u,v}(ca)$,
\item
$Y_{u,vc}(ca)=Y_{uc^2,v}(a)$,
\item
$Y_{u,v}(a)X_{u',v}(a)=Y_{u+u',v}(a)$,
\item
$g\,Y_{wb,v}(a)g\inv=Y_{\phi(g)wb,\phi(g)^*v}(a)$.
\end{enumerate}
\end{lm}

Finally, it only remains to show that for a ``good'' pair $(u,\,v)$ elements $X$ and $Y$ coincide.

\begin{lm}
\label{x=y}
Consider $w$, $u$, $z$, $v$, $x$, $y\in B^n$ $a$, $r\in B$ such that $w^tu=a$, $z^tv=a$, $x^ty=a$ and pairs $(w,\,u)$, $(z,\,v)$ and $(x,\,y)$ are mutually orthogonal. Then 
$$
X_{u,v}(ra^3)=Y_{u,v}(ra^3).
$$
\end{lm}

\begin{proof}
On one hand,
\begin{multline*}
[Y_{x,-v}(ra),\,X_{u,y}(a)]=X_{t(x,-vra)u,t(x,-vra)^*y}(a)X_{u,-y}(a)=\\
=X_{u,y+vra^2}(a)X_{u,-y}(a)=X_{u,v}(ra^3).
\end{multline*}
On the other,
\begin{multline*}
[Y_{x,-v}(ra),\,X_{u,y}(a)]=Y_{x,-v}(ra)Y_{t(u,ya)x,t(u,ya)^*v}(ra)=\\
=Y_{-x,v}(ra)Y_{x+ua^2,v}(ra)=Y_{u,v}(ra^3).
\end{multline*}
\end{proof}

Now, we are ready to construct a map 
$
\mathrm T\colon\St(n,\,B_a,\,I)\rightarrow\St(n,\,B).
$

\begin{proof}[Proof of Theorem~\ref{a3map}]
For $u=Me_1$, $M\in\E(n,\,B_a)$, $v\in I^n$, $u^tv=0$ denote $w=M^*e_1$ (then $wu=1$) and choose $m\in\mathbb N$ such that some $\tilde w$, $\tilde u\in B^n$ localise to $wa^m$ and $ua^m$, $\tilde u^tv=0$ and $\tilde w^t\tilde u=a^{2m}$. Now set $\mathrm T(F(u,\,v))=X_{\tilde u,v/a^{3m}}(a^{2m})$. Lemma~\ref{xproperties}\,a) and b) garantees that this definition does not depend on the choice of $m$ and the lifts $\tilde u$ and $\tilde w$. Similarly, set $\mathrm T(S(u,\,v))=Y_{u/a^{3m},\tilde v}(a^{2m})$. These elements satisfy R1--R3 by Lemmas~\ref{xproperties} and~\ref{yproperties}. For $u=Me_1$, $v=M^*e_2$, $M\in\E(n,\,B_a)$ consider $w=M^*e_1$, $z=Me_2$, $x=Me_3$ and $y=M^*e_3$. Then, multiply these element on appropriate power of $a$, take their lifts to $B$ and apply Lemma~\ref{x=y} to get R4. So that, the map $\mathrm T$ is well-defined.

Commutativity of the diagram follows directly from the definitions of elements $X(u,\,v)$, $Y(u,\,v)$ and $X_{u,v}(a)$, $Y_{u,v}(a)$ and the possibility to redistribute the powers of $a$ after localisation.
\end{proof}

The next section is devoted to the proof of Theorem~\ref{a3map} for Steinberg groups corresponding to an arbitrary simply-laced root system of rank $l\geq3$. To do so, we will ``glue'' maps $T$ constructed in the present section for systems of type $\mathrm A_3$. All main results of the present paper follow from the existence of such a map.

\printbibliography

\end{document}

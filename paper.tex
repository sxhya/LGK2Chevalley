\documentclass[12pt]{amsart}
\usepackage{amscd, amsmath, amssymb, amsthm, amsfonts, amstext, verbatim, enumitem, graphicx, mathtools, xfrac, tikz-cd, nameref, thmtools, hyperref}
\usepackage[utf8]{inputenc}
\usepackage[backend=biber, citestyle=numeric-comp, natbib=true, sortlocale=en_US, url=false, doi=false, eprint=true, maxbibnames=4]{biblatex}            
\addbibresource{paper.bib}
\usepackage[activate={true,nocompatibility}]{microtype}
\usepackage[capitalize]{cleveref}
            
\newlist{thmlist}{enumerate}{1} \setlist[thmlist]{label=(\roman{thmlisti}), ref=\thethm.(\roman{thmlisti}),noitemsep} \Crefname{thmlisti}{Theorem}{Theorems}
\newlist{lemlist}{enumerate}{1} \setlist[lemlist]{label=(\roman{lemlisti}), ref=\thelemma.(\roman{lemlisti}),noitemsep} \Crefname{lemlisti}{Lemma}{Lemmas}

\theoremstyle{plain} \declaretheorem[name=Theorem, Refname={Theorem,Theorems}]{thm} \Crefname{thm}{Theorem}{Theorems}
\numberwithin{equation}{section}

\newtheorem{lemma}{Lemma} \numberwithin{lemma}{section} \Crefname{lemma}{Lemma}{Lemmas}
\newtheorem{cor}[lemma]{Corollary} \Crefname{cor}{Corollary}{Corollaries}
\newtheorem{prop}[lemma]{Proposition} \Crefname{prop}{Proposition}{Propositions}
\newtheorem*{thm*}{Theorem}
\newtheorem*{lemma*}{Lemma}
\theoremstyle{definition} \newtheorem{dfn}[lemma]{Definition} \Crefname{dfn}{Definition}{Definitions}
\newtheorem{example}[lemma]{Example} \Crefname{example}{Example}{Examples}
\theoremstyle{remark} \newtheorem{rem}[lemma]{Remark} \Crefname{rem}{Remark}{Remarks}

%Page layout
\textwidth 16cm 
\textheight 22cm 
\headheight 0.5cm 
\evensidemargin 0.3cm 
\oddsidemargin 0.2cm

\newcommand{\catname}[1]{{\normalfont\textbf{#1}}}
\newcommand{\Ker}{\operatorname{Ker}}
\newcommand{\Cent}{\operatorname{Cent}}
\newcommand{\Img}{\operatorname{Im}}
\newcommand{\Coker}{\operatorname{Coker}}
\newcommand{\coker}{\operatorname{coker}}
\newcommand{\K}{\operatorname{\mathrm{K}}}
\newcommand{\GG}{\operatorname{\mathrm{G}}}
\newcommand{\St}{\operatorname{\mathrm{St}}}
\newcommand{\StU}{\operatorname{\hat{\mathrm{U}}}}
\newcommand{\G}{\operatorname{\mathrm{G}}}
\newcommand{\E}{\operatorname{\mathrm{E}}}
\newcommand{\M}{\operatorname{\mathrm{M}}}
\newcommand{\sr}{\operatorname{\mathrm{sr}}}
\newcommand{\Um}{\operatorname{\mathrm{Um}}}
\newcommand{\rk}{\operatorname{\mathrm{rk}}}
\newcommand{\rA}{\mathsf{A}}
\newcommand{\rB}{\mathsf{B}}
\newcommand{\rC}{\mathsf{C}} 
\newcommand{\rD}{\mathsf{D}} 
\newcommand{\rE}{\mathsf{E}}
\newcommand{\HH}{\mathrm{H}}
\newcommand{\eval}[4]{ev \textstyle \left[\frac{#2[#1] \rightarrow #3}{#1 \mapsto #4}\right]}
\newcommand{\ev}[3]{\eval{t}{#1}{#2}{#3}}

\newcommand\restr[2]{{\left.\kern-\nulldelimiterspace #1 \vphantom{\big|}\right|_{#2}}}
\newcommand*\MapsTo{\xrightarrow[\raisebox{0.25 em}{\smash{\ensuremath{\sim}}}]{}}  

\title{Local-global principle for $\K_2$ of Chevalley groups}
\keywords {Chevalley groups, Steinberg groups, $K_2$-functor. {\em Mathematical Subject Classification (2010):} 19C09; 20G35}
\author {Andrei Lavrenov, Sergey Sinchuk}
\email {avlavrenov {\it at} gmail.com, sinchukss {\it at} gmail.com}
\date {\today}

\begin{document}

\begin{abstract} To be written. \end{abstract}

\maketitle

\section {Introduction}\label{intro}
 
\subsection{Acknowledgements}
The authors of the present paper acknowledges financial support from Russian Science Foundation grant 14-11-00297.

\section {Preliminaries}\label{prelim}

\section {Main results}

\subsection{Relative Steinberg group in rank $3$ and Tulenbaev's section map} 
The main goal of this subsection is to construct Tulenbaev's section map from~\cite[\S~2]{T} for
the linear Steinberg group of rank $3$. Combining this result with the amalgamation theorem
of the the second named author (see~\cite[Theorem~9]{S15}) we deduce centrality for the orthogonal
$\K_2$ in the next subsection.

First of all, we will need to give yet another definition of the relative linear Steinberg group 
(cf. \cite[Definitions~3.3 and 3.7]{S15}).

\begin{dfn}
 The relative Steinberg group $\St^*(n,R, I)$ is the group defined by the following two
 families generators and four families of relations.
 \begin{itemize}
  \item Generators:
  \begin{enumerate}
  \item $X^1(u, v)$, where $u \in \E(n,R) \cdot e_1$, $v\ \in I^n$ such that $v^t \cdot u = 0$;
  \item $X^2(u, v)$, where $u \in I^n$, $v \in \E(n,R) \cdot e_1$ such that $v^t \cdot u = 0$.
 \end{enumerate}
  Notice that the canonical image in $\E(n, R)$ of both $X^1(u, v)$ and $X^2(u, v)$ equals $T(u, v) = e_n + u \cdot v^T$.
  \item Relations:
  \begin{enumerate}
  \item $X^1(u, v) \cdot X^1(u, w) = X^1(u, v+w)$, $u \in \E(n,R) \cdot e_1$, $v, w \in I^n$;
  \item $X^2(u, v) \cdot X^2(w, v) = X^2(u+w, v)$, $u, w \in I^n$, $v \in \E(n,R) \cdot e_1$;
  \item ${}^{X^\sigma(u^2, v^2)} \! X^\tau(u^1, v^1) = X^\tau(T(u^2, v^2) \cdot u^1, T(v^2, u^2)^{-1} \cdot v^1)$, $\sigma, \tau = 1..2$;
  \item $X^1(g \cdot e_1, g^* \cdot be_2) = X^2(g \cdot be_1, g^* \cdot e_2)$ where $g^* = {g^t}^{-1}$ is the contragradient matrix and $b\in I$.
 \end{enumerate}
 \end{itemize}
\end{dfn}
It is easy to check the correctness of the above definition.

\begin{lemma}
 The groups $\St^*(n, R, I)$ and $\St(n, R, I)$ are isomorphic.
\end{lemma}
\begin{proof}
 TODO: 
\end{proof}

The next step of the proof is to is construct certain elements in $\St(n, R)$ similar to Tulenbaev's elements $X_{u,v}(a)$ see~\cite[\S~1]{T}.

Assume that $n \geq 4$. For the rest of this section $a$ denotes a nonnilpotent element of $R$.

Let $v\in R^n$ be a column.
A column $w\in R^n$ is called \emph{$v$-decomposable} if it can be presented as a finite sum $w = \sum\limits_{i=1}^n w^i$ of columns such that each $w^i$ 
has at least two zero entries. 
Denote by $D(v)$ the submodule of $R^n$ consisting of all $v$-decomposable columns.
%For a column $v\in R^n$ denote by $I(v)$ the ideal of $R$ spanned by coordinates $v_1,\ldots, v_n$.

Let $u,v,w\in R^n$ be columns such that $w^tv=0$.
It is easy to check that $(uv)\cdot w = \sum_{i<j}w_{ij}$, where $w_{ij} = (w_iu_j - w_ju_i)(v_je_i - v_ie_j)\in{}\!A^n$ (cf.~\cite[Lemma~3.2]{Ka}).
Such decomposition is called the {\it canonical} decomposition of $(uv)\cdot w$.
In particular, this shows that row $a\cdot w$ is always $v$-decomposable for $a\in I(v)$, i.e. $I(v) \cdot O(v) \subseteq D(v)$.
It is also obvious that $O(v)\subseteq O(bv)$, $D(v)\subseteq D(bv)$ for $b \in R$.

Let $v$ be a (not neccessarily unimodular) column of height $n$, let $w$ be a $v$-decomposable row and $a$ be an element of $I(v)$.
In this context Tulenbaev defines elements $X_{v,w}(a)$ of the Steinberg group $\St(n, A)$, $n\geq 4$ (see~\cite[\S~1]{T}).
Tulenbaev requires $a \in I(v)$ in order to prove correctness of his definition.

Let us list certain useful properties of these elements.
\begin{lemma}\label{T13} Let $n\geq 4$. Elements $X_{v,w}(a)$ satisfy the following relations:
\begin{enumerate}
 \item $X_{v, w}(ab) = X_{v, bw}(a)$ for $w\in D(v)$, $b\in R$, $a\in I(v)$. If moreover $a\in b \cdot I(v)$ then $X_{v, w}(ab) = X_{bv, w}(a)$.
 \item $X_{v,w_1 + w_2}(a) = X_{v, w_1}(a) X_{v, w_2}(a)$, $w_1, w_2 \in D(v)$, $a\in I(v)$.
 \item \label{T133} For $g\in\St(n, A)$,  $w\in O(v)$, $a,b\in I(v)$  one has  $g^{-1}\cdot X_{v, bw}(a)\cdot g = X_{g^{-1}v, bwg}(a)$
 \item Let $a$ be a nonnilpotent element of $A$.
       For $n\geq 5$ let $(v_1,v_2)\in M_{n\times 2}(A)$ be a matrix such that its image in $M_{n\times 2}(A_a)$ is unimodular.
       Let $w \in {}^n\!A$ be a row such that $wv_1=wv_2=0$. Then for sufficiently large natural numbers $k, m$ one has
       $$X_{v_1 + bv_2, a^kw}(a^m) = X_{v_1,a^kw}(a^m)\cdot X_{v_2,ba^kw}(a^m),\ b\in A.$$
\end{enumerate}\end{lemma}



\begin{lemma}
 Let $u, v \in R^n$ be columns such that $v^t \cdot u = 0$ and there exists $w = w(u) \in R^n$ satisfying $w^t \cdot u = a^N$ for some natural number $N$.
 In this context one can define certain elements $Z^{1}(u,v), Z^{2}(v, u)$ of $\St(n, R)$. These elements satisfy the following relations:
 \begin{itemize}
  \item $Z^{1}(u, v_1 + v_2) = Z^{1}(u, v_1) \cdot Z^{1}(u, v_2)$;
  \item $Z^{2}(v_1 + v_2, u) = Z^{2}(v_1, u) \cdot Z^{2}(v_2, u)$;
  \item ${}^{g}\! Z^{\tau}(u, v) = Z^{\tau}(g \cdot u, g^* \cdot v)$, $\tau = 1..2$, $g \in \St(n, R)$.
 \end{itemize}
\end{lemma}
\begin{proof} Both the construction of $Z^i$ and the relations for $Z^i$ are, in essence, just~\cite[Lemma~1.3]{T} applied twice. \end{proof}

\begin{lemma}
 For $u_1, u_2 \in R^n$ be columns such that $\overline{u_1} = g \cdot e_1$, $\overline{u_2} = g^* \cdot e_2$ for some $g \in \E(n, R_a)$.
 Then for any $b\in I$ and $g',g'' \in M(n, R)$ such that $\overline{g'} = g$, $\overline{g''} = g^*$ one has 
 $$Z^1(g' \cdot e_1, g'' \cdot be_2) = Z^2(g' \cdot be_1, g'' \cdot e_2).$$
\end{lemma}


%TODO: Why we can assume u_1^t \cdot u_2 = 0$?
Now let $g \in \E(n, R_a)$ be such that $g(e_1, e_2) = (\overline{u_1}, \overline{u_2})$ for some $u_1, u_2\in R^n$, $u_1^t \cdot u_2 = 0$.
It is easy to find $w_1, w_2 \in R^n$ such that $w_1^t \cdot u_1 = a^{N_1}$, $w_2^t \cdot u_2 = a^{N_2}$.


$Z^1(u, v) = Z^2(v, u)$.


for each $(u,v) \in \E_n(R) (e_1, e_2)$ we can find $y$ such that $(u,v,y) = g (e_1, e_2, e_3)$.

$w^tu = a^N$, $z^Tv = a^N$, $x^T y = a^N$ + 6 additional orthogonality relations

Computing $g = [Z^{1, N}(u xba^{2N}), Z^{2, N}(ua^{2N},v)]$ 
$$ g = Z^{2, N}(ya^{2N} + uba^{5N},v) {Z^{2, N}(ya^{2N}, v)}^{-1} = Z^{2, N}(uba^{5N},v) = Z^2(uc, v)$$
$$ g = ... = Z^1(u, vc)$$

\subsection{Centrality of orthogonal $K_2$.} 

\printbibliography

\end{document}

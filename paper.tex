\documentclass[11pt]{amsart}
\usepackage{amscd, amsmath, amssymb, amsthm, amsfonts, amstext, verbatim, enumitem, mathtools, xfrac, nameref, thmtools, hyperref}
\usepackage[utf8]{inputenc}
\usepackage[backend=biber, citestyle=numeric-comp, natbib=true, sortlocale=en_US, url=false, doi=false, eprint=true, maxbibnames=4]{biblatex}            
\usepackage[activate={true,nocompatibility}]{microtype}
\usepackage[capitalize]{cleveref}
\usepackage[english]{babel} 
\usepackage{stmaryrd}
\usepackage[matrix,arrow,curve]{xy}

%Bibliography
\renewbibmacro{in:}{\ifentrytype{article}{}{\printtext{\bibstring{in}\intitlepunct}}}
\newbibmacro{string+doi}[1]{\iffieldundef{doi}{\iffieldundef{url}{#1}{\href{\thefield{url}}{#1}}}{\href{http://dx.doi.org/\thefield{doi}}{#1}}}
\DeclareFieldFormat{title}{\usebibmacro{string+doi}{\mkbibemph{#1}}}
\DeclareFieldFormat[article, inproceedings, inbook, thesis]{title}{\usebibmacro{string+doi}{\mkbibquote{#1}}}
\addbibresource{paper.bib}
            
\newlist{thmlist}{enumerate}{1} \setlist[thmlist]{label=(\roman{thmlisti}), ref=\thethm.(\roman{thmlisti}),noitemsep} \Crefname{thmlisti}{Theorem}{Theorems}
\newlist{lemlist}{enumerate}{1} \setlist[lemlist]{label=(\roman{lemlisti}), ref=\thelemma.(\roman{lemlisti}),noitemsep} \Crefname{lemlisti}{Lemma}{Lemmas}

\theoremstyle{plain} \declaretheorem[name=Theorem, Refname={Theorem,Theorems}]{tm} \Crefname{tm}{Theorem}{Theorems}
\numberwithin{equation}{section}

\newtheorem{lm}{Lemma} \numberwithin{lm}{section} \Crefname{lm}{Lemma}{Lemmas}
\newtheorem{cl}[lm]{Corollary} \Crefname{cl}{Corollary}{Corollaries}
\newtheorem{prop}[lm]{Proposition} \Crefname{prop}{Proposition}{Propositions}
\newtheorem*{tm*}{Theorem}
\newtheorem*{lm*}{Lemma}
\theoremstyle{definition} \newtheorem{df}[lm]{Definition} \Crefname{df}{Definition}{Definitions}
\newtheorem{example}[lm]{Example} \Crefname{example}{Example}{Examples}
\theoremstyle{remark} \newtheorem{rk}[lm]{Remark} \Crefname{rk}{Remark}{Remarks}
\newtheorem*{mtm}{Main Theorem}
\newtheorem*{atm}{Another Presentation}
\newlist{lmlist}{enumerate}{1} \setlist[lmlist]{label={\rm(\roman{lmlisti})}, ref=\thelm.(\roman{lmlisti}),noitemsep} \Crefname{lmlisti}{Lemma}{Lemmas}


%Page layout
\textwidth 16cm 
\textheight 22cm 
\headheight 0.5cm 
\evensidemargin 0.3cm 
\oddsidemargin 0.2cm

\newcommand{\Max}{\mathop{\mathrm{Max}}\nolimits}
\newcommand{\proj}{\mathop{\mathrm{proj}}\nolimits}
\newcommand{\Card}{\mathop{\mathrm{Card}}\nolimits}
\newcommand{\Ker}{\mathop{\mathrm{Ker}}\nolimits}
\newcommand{\Cent}{\mathop{\mathrm{Cent}}\nolimits}
\newcommand{\E}{{\mathrm{E}}}
\newcommand{\GG}{{\mathrm{G}}}
\newcommand{\Um}{\mathop{\mathrm{Um}}\nolimits}
\newcommand{\St}{\mathop{\mathrm{St}}\nolimits}
\newcommand{\Sp}{\mathop{\mathrm{Sp}}\nolimits}
\newcommand{\Ep}{\mathop{\mathrm{Ep}}\nolimits}
\newcommand{\GL}{\mathop{\mathrm{GL}}\nolimits}
\newcommand{\SL}{\mathop{\mathrm{SL}}\nolimits}
\newcommand{\Kt}{\mathop{\mathrm{K_2}}\nolimits}
\newcommand{\Ko}{\mathop{\mathrm{K_1}}\nolimits}
\newcommand{\Ho}{\mathop{\mathrm{H_1}}\nolimits}
\newcommand{\Ht}{\mathop{\mathrm{H_2}}\nolimits}
\newcommand{\epi}{\twoheadrightarrow}
\newcommand{\sgn}{\mathrm{sgn}}
\newcommand{\eps}[1]{\varepsilon_{#1}}
\newcommand{\sign}[1]{\mathrm{sign}(#1)}
\newcommand{\lan}{\langle}
\newcommand{\ran}{\rangle}
\newcommand{\inv}{^{-1}}
\newcommand{\ur}[1]{\!\,^{(#1)}U_1}
\newcommand{\ps}[1]{\!\,^{(#1)}\!P_1}
\newcommand{\ls}[1]{\!\,^{(#1)}\!L_1}
\newcommand{\rA}{\mathsf{A}}
\newcommand{\rB}{\mathsf{B}}
\newcommand{\rC}{\mathsf{C}} 
\newcommand{\rD}{\mathsf{D}} 
\newcommand{\rE}{\mathsf{E}}
\newcommand{\rF}{\mathsf{F}}
\newcommand{\rG}{\mathsf{G}}

\renewcommand{\labelenumi}{\theenumi{\rm)}}
\renewcommand{\theenumi}{\alph{enumi}}

\newcommand\restr[2]{{\left.\kern-\nulldelimiterspace #1 \vphantom{\big|}\right|_{#2}}}
\newcommand*\MapsTo{\xrightarrow[\raisebox{0.25 em}{\smash{\ensuremath{\sim}}}]{}}  

\title{On centrality of orthogonal $\Kt$}
\keywords {Steinberg groups, $K_2$-functor. {\em Mathematical Subject Classification (2010):} 19C09}

\author {Andrei Lavrenov}
\email {avlavrenov {\it at} gmail.com}

\author{Sergey Sinchuk}
\email{sinchukss {\it at} gmail.com}

\date {\today}
\thanks{Authors of the present paper acknowledge the financial support from Russian Science Foundation grant 14-11-00297}

\begin{document}

\begin{abstract} We give a short uniform proof of centrality of $\mathrm K_2(\Phi, R)$ for all simply-laced root systems $\Phi$ of rank $\geq 3$.
\end{abstract}

\maketitle

%\begin{thebibliography}{99}
%\bibitem{Lav2} A.~Lavrenov, A local--global principle for symplectic $\mathrm K_2$, {\it to appear}.
%\bibitem{Rao} *Statement of Ravi Rao's hypothesis
%\bibitem{Swa1} *Swan's relative Steinberg group 1
%\bibitem{Swa2} *Swan's relative Steinberg group 2
%\bibitem{Swa3} *Swan's relative Steinberg group 3
%\bibitem{Tul0} *Tulenbaev's paper on almost commutative rings (cite somewhere)
%\end{thebibliography}

\section*{Inroduction}
The main goal of this paper is to establish the local-global principle and centrality of orthogonal $\Kt$ for arbitrary commutative ring $R$.

Recall that for each reduced root system $\Phi$ and a commutative ring $R$ one can associate a split simple group $\GG(\Phi, R)$ called the \emph{Chevalley group}.
For $\Phi$ of rank $\geq 2$ the abstract group $\GG(\Phi, R)$ contains a large normal subgroup $\E(\Phi, R)$ generated by the elementary root unipotents $t_\alpha(\xi)$, $\xi\in R$, $\alpha\in \Phi$.
The quotient group functor $\Ko(\Phi, R)=\GG(\Phi, R)/\E(\Phi, R)$ is called the \emph{unstable $\Ko$-functor}.
This functor shares many properties with its stable counterpart $\mathrm{SK}_1(R) = \lim\limits_{l\to\infty}\Ko(\rA_\ell, R)$ which, in turn, is an essential direct summand of $\Ko(R)$.

Similarly to the definition of the algebraic $\Kt$-functor, one can define $\Kt(\Phi, R)$ as the kernel of the canonical projection $\phi\colon\St(\Phi, R)\twoheadrightarrow \E(\Phi, R)$.
Here $\St(\Phi, R)$ stands for the \emph{Steinberg group} of type $\Phi$, i.\,e. the group defined by formal generators $x_\alpha(\xi)$ modeling root unipotents $t_\alpha(\xi)$
and the explicit list of relations called Steinberg relations (see~\cref{sec:prelim}). 

It is classically known that $\Kt(\Phi, R)$ is a central subgroup of $\St(\Phi, R)$ if $R$ is commutative local ring and $\Phi$ has rank $\geq 2$ (see \cite{St73}).
However, to prove that centrality of $\Kt$ holds for \emph{arbitrary} $R$ is much harder.
The first result in this direction was obtained in 1977 by W. van~der~Kallen for $\Phi=\rA_\ell$, $\ell\geq 3$ using the technique called ``another presentation''.
Using a similar approach, the first-named author has established centrality of $\Kt(\rC_\ell, R)$ for $\ell\geq 3$, see \cite{Lav}.
In turn, the key result of this paper is the following theorem.
\begin{tm}\label{tm:centrality}  Let $R$ be an arbitrary commutative ring and $\ell\geq 4$ then 
the group $\Kt(\rD_\ell, R)$ is contained in the centre of $\St(\rD_\ell, R)$. \end{tm}
Notice that the analogous statement in the case $\Phi=\rE_\ell$, $\ell=6,7,8$ has been demonstrated by the second named author in \cite{SCh}.
\cref{tm:centrality} is essentially a corollary of the following local-global principle.
\begin{tm}\label{tm:lg-principle} Let $\Phi$ be an irreducible simply-laced root system of rank $\geq 3$ and $R$ be arbitrary commutative ring.
Then an element $g\in\St(\Phi,\,R[X])$ satisfying $g(0)=1\in\St(\Phi,\,R)$ is trivial in $\St(\Phi,\,R[X])$
if and only if the elements $g_M\in\St(\Phi,\,R_M[X])$ are trivial for all maximal ideals $M\triangleleft R$. \end{tm} 

The special case $\Phi=\rA_\ell$, $\ell\geq 4$ of \cref{tm:lg-principle} was demonstrated for the first time by Tulenbaev, see~\cite[Theorem~2.1]{Tul},
while similar assertions for $\Phi=\rC_\ell$, $\ell\geq 3$ and $\Phi=\rE_\ell$, $\ell=6,7,8$ have been established in \cite{Lav2} and \cite{SCh}, respectively.
The special cases $\Phi=\rA_3$ and $\Phi=\rD_\ell$, $\ell\geq 4$ of \cref{tm:lg-principle} are new and have not been known before.

There are known counterexamples to centrality of $\Kt$ in the rank $2$ case (see~\cite{W}).
This shows that the assumption on the rank of $\Phi$ in \cref{tm:lg-principle} is strict.

Compare \cref{tm:lg-principle} with the local-global principle for $\Ko$ which states that an element of the congruence subgroup $g\in \GG(\Phi,\,R[X], XR[X])$ is elementary
(i.\,e. lies in $\E(\Phi, R[X])$) if and only if $g_M=1\in\E(\Phi,\,R_M)$ for all maximal ideals $M \triangleleft R$.

Our proof of \cref{tm:lg-principle} follows the same lines as in \cite[Theorem~1]{SCh}.
More concretely, using the same patching technique as in \cite{SCh} we reduce the problem to the case $\Phi=\rA_3$.
Then, we formulate and prove a new presentation of the linear group $\St(4, R)$ and then utilize it to reprove 
Tulenbaev's key lemma (cf. \cite[Lemma~2.3]{Tul}) in the rank $3$ case.

The rest of the paper is organised as follows. 
In \cref{sec:yap} we formulate ``yet another presentation'' for the rank $3$ Steinberg groups.
%It mimics the standart vector representation of $\SL(n,\,R)$, therefore in the first section we use representations and corresponding vector notation. 
Finally, in \cref{sec:patching} we prove our main results.
%There we switch to the root notation.

\section{Preliminaries} \label{sec:prelim}
Throughout this paper $R$ denotes an associative commutative ring with identity.
All commutators are left-normed, i.\,e. $[x,\,y]=xyx\inv y\inv$. 

%Tulenbaev used {\it another presentation} for a relative Steinberg group to prove a local--global
%principle for a linear Steinberg group~\cite{Tul}. 
%This relative another presentation was inspired by the {\it absolute} one of van der Kallen~\cite{vdK}. 
%However, Tulenbaev's presentation can only be used for $\St(l+1,\,R)=\St(\mathrm A_l,\,R)$, $l\geq4$.

%Tulenbaev's result can be applied to the subsystems of type $\mathrm A_4$ inside $\mathrm E_l$ and ``glued'' 
% to get a local--global principle for Steinberg groups of type $\mathrm E_l$~\cite{SCh}. 
%To use this glueing method for groups of type $\mathrm D_l$ as well one has to work with subsystems 
%of type $\mathrm A_3$ instead, and thus extend Tulenbaev's result to the case of $\mathrm A_3$.

%Recently, the local--global principle was obtained for Steinberg groups of type $\mathrm C_l$ starting from rank 3~\cite{Lav2}. 
%That is another motivation to consider the linear Steinberg group in rank 3.

We denote by $R^n$ the free $R$-module with basis $e_1,\ldots,e_n$ and by $\Um(n,\,R)$ the subset of unimodular columns $v\in R^n$ 
whose entries generate $R$ as an ideal.

As usual, $\E(n, R)=\E(\rA_{n-1}, R)$ denotes the elementary group, i.\,e. the subgroup of the general linear group $\GL(n, R)$ generated by transvections
$t_{ij}(r)=1+r \cdot e_{ij}$, $1\leq i\neq j\leq n$, $r\in R$,
where 1 stands for the identity matrix and $e_{ij}$ for the matrix unit.

The Steinberg group $\St(n, R) = \St(\rA_{n-1}, R)$ is the group presented by generators
$x_{ij}(r)$, $1\leq i\neq j\leq n$, $r\in R\}$ and relations
\setcounter{equation}{0}
\renewcommand{\theequation}{S\arabic{equation}}
\begin{align}
&x_{ij}(r)x_{ij}(s)=x_{ij}(r+s), \label{add0}\\
&[x_{ij}(r),\,x_{hk}(s)]=1,\text{ for }h\neq j,\ k\neq i, \label{ccf1}\\
&[x_{ij}(r),\,x_{jk}(s)]=x_{ik}(rs) \label{ccf2}.
\end{align}
The natural projection $\phi\colon\St(n,\,R)\rightarrow\E(n,\,R)$ sends $x_{ij}(r)$ to $t_{ij}(r)$.

In \cite{Sus} Suslin showed for $n\geq 3$ that the elementary group $\E(n, R)$ coincides with the subgroup of $\GL(n,\,R)$ generated by matrices
of the form $t(u,\,v)=1+uv^t$ where $u\in\Um(n,\,R)$, $v\in R^n$ and $u^tv=0$. Here $u^t$ stands for the transpose of $u$.
This result clearly implies that $\E(n,\,R)$ is normal inside $\GL(n,\,R)$.
Afterwards, W.~van~der~Kallen developed Suslin's ideas and showed for $n\geq4$ that the Steinberg group $\St(n,\,R)$ is isomorphic to the group presented by generators
$$\{X(u,\,v)\mid u\in\Um(n,\,R),\ v\in R^n,\ u^tv=0\}$$ and the following list of relations (see~\cite[Theorem~1]{vdK}):
\setcounter{equation}{0} \renewcommand{\theequation}{K\arabic{equation}}
\begin{align}
&X(u,\,v)X(u,\,w)=X(u,\,v+w), \label{add1} \\
&X(u,\,v)X(u',\,v')X(u,\,v)\inv=X(t(u,\,v)u',\,t(v,\,u)\inv v'). \label{conj1}
\end{align}
This presentation clearly implies that $\phi\colon\St(n,\,R)\rightarrow\E(n,\,R)$ is a central extension.
Notice that our notation for the generators of $\St(n, R)$ slightly differs from that of~\cite{vdK}, e.\,g. we parametrize elements $X(u, v)$ with two columns
rather than with one column and one row. Our generator $X(u,v)$ corresponds to van der Kallen's generator $(u, v^t)$.

Recall that an ideal $I\trianglelefteq R$ is called a \emph{splitting ideal} if the canonical projection $R \twoheadrightarrow R/I$ splits as a unital ring morphism, i.\,e. one has $\pi\sigma=1$.
$$\xymatrix{0\ar@<0.1ex>[r]^{}&I\ar@<0.1ex>[r]^{}&R\ar@<0.5ex>[r]^{\pi}&R/I\ar@<0.5ex>[l]^{\sigma}\ar@<0.1ex>[r]^{}&0}$$
For a splitting ideal $I\trianglelefteq R$ the \emph{relative Steinberg group} $\St(n,\,R,\,I) = \St(\rA_{n-1}, R, I)$ can be defined as the kernel $\Ker\big(\St(n,\,R)\epi\St(n,\,R/I)\big)$.

For a general $I$ the relative Steinberg group $\St(\Phi, R, I)$ is no more a subgroup of $\St(\Phi, R)$ but rather is its central extension.
For the purposes of the present text it suffices to consider only relative Steinberg groups corresponding to splitting ideals $I$.
For more information regarding the general case we refer the reader to \cite[Section~3]{SCh}.

\section{Yet another presentation for $\St(4, R)$.} \label{sec:yap}
The key ingredient in Tulenbaev's proof of the local-global principle is the construction of the homomorphism $T$ which fits into the following commutative diagram.
$$\xymatrix{\St(n,\,R\ltimes XR_a[X],\,XR_a[X]) \ar@{^{(}->}[r] \ar[d] & \St(n, R \ltimes XR_a[X]) \ar[d]_-{\lambda_a^*} \\
             \St(n,\,R_a[X],\,XR_a[X]) \ar@{-->}[ru]_-{T} \ar@{^{(}->}[r] & \St(n, R_a[X])}$$
Here $n\geq5$, $a$ is arbitrary nonnilpotent element of $R$ and $\lambda_a^*$ is the morphism induced by the morphism $\lambda_a\colon R \to R_a$ of principal localisation at $a$.
Notice that $XR_a[X]$ is a splitting ideal for both $R_a[X]$ and $R \ltimes XR_a[X]$.
It is clear that the image of $T$ is contained in the image of the top arrow, therefore this implies that the relative Steinberg groups in the left-hand side of the diagram are isomoprhic.
%This $T$ is an inverse to a map induced by localisation, so that constructing $\mathrm T$ may be treated
%as a proof of excision property for Steinberg group in this very particular situation.

%To prove the local--global principle for Steinberg groups of type $\mathrm D_l$ and centrality 
%of $\mathrm K_2(\mathrm D_l,\,R)$ we need such a map for $n=4$ (i.e., for the group of type $\mathrm A_3$).

In order to construct the arrow $T$ Tulenbaev uses the following presentation for the relative Steinberg group, see~\cite[Proposition~1.6]{Tul}. 
%Assume that $I\trianglelefteq R$ such that a short exact sequence
%\renewcommand{\theequation}{$\dagger$}
\begin{prop}\label{prop:TulPres}
For a splitting ideal $I$ and $n\geq 4$ the group $\St(n,\,R,\,I)$ is isomorphic to the group defined by generators
$$\{X(u,\,v)\mid u\in\E(n,\,R)e_1,\ v\in I^n,\ u^tv=0\}$$ and the following set of relations
\setcounter{equation}{0}
\renewcommand{\theequation}{T\arabic{equation}}
\begin{align}
&X(u,\,v)X(u,\,w)=X(u,\,v+w), \label{add2}\\
&X(u,\,v)X(u',\,v')X(u,\,v)\inv=X(t(u,\,v)u',\,t(v,\,u)\inv v'), \label{conj2}  \\
&X(ur+w,\,v)=X(u,\,vr)X(w,\,v)\,\text{ for }r\in R,\ (u,\,w)\in\Um_{n\times2}(R) \label{add3}.
\end{align}
Moreover, the group $\St(n, R)$ decomposes into the semidirect product $\St(n,\,R, I)\rtimes \St(n,\,R/I)$.
\end{prop}
Here $\Um_{n\times2}(R)$ denotes the set of $n\times2$ \emph{unimodular matrices}, i.\,e. matrices $M$ such that there exists a $2\times n$ matrix $N$ satisfying $NM=\begin{pmatrix}1&0\\0&1\end{pmatrix}$.
In the definition of the semidirect product the group $\St(n, R/I)$ acts on $\St(n, R, I)$ by conjugation via the splitting map, i.\,e. $g \cdot X(u,\,v) = X(\phi(s(g))u, \phi(s(g))^*v)$.

\begin{proof} For the sake of completeness we present below a more detailed version of Tulenbaev's proof.
Tulenbaev constructs a pair of mutually inverse maps
$$\xymatrix{ \St(n, R, I) \rtimes \St(n, R/I) \ar@<0.6ex>[r]^-\varphi &  \ar[l]^-\psi \St(n, R).}$$

Define the map $\psi$ by $\psi(x_{ij}(\xi)) = (X(e_i, (\xi - \pi(\xi)e_j),\ x_{ij}(\pi(\xi)))$. 
We have to check that $\psi$ preserves the relations \eqref{add0}--\eqref{ccf2}.
For example, let us check \eqref{ccf1}. 
First of all, recall that in any semidirect product of groups one can compute the commutator of two elements as follows:
$$[(a, b), (c, d)] = (a \cdot {}^bc \cdot {}^{bdb^{-1}}a^{-1} \cdot {}^{[b, d]}c^{-1} , [b, d]).$$
For $\xi\in R$ set $\xi' = \xi - \pi(\xi)$.
Specializing $a = X(e_i, \xi' e_j)$, $b = x_{ij}(\pi(\xi))$, $c = X(e_j, \eta'e_k)$, $d = x_{jk}(\pi(\eta))$ and using the definition of the conjugation action 
one computes the commutator $[\psi(x_{ij}(\xi), \psi(x_{jk}(\eta)]$ as follows:
$$\left(X(e_i, \xi' e_j) \cdot X(e_j + \pi(\xi)e_i, \eta'e_k) \cdot X(e_i, \pi(\eta)\xi' e_k - \xi' e_j) \cdot X(e_j, -\eta'e_k),\ x_{ik}(\pi(\xi\eta))\right).$$
After applying additivity relations \eqref{add2}, \eqref{add3} and moving the factor $X(e_i, -\xi' e_j)$ to the left hand side of the formula by means of relation \eqref{conj2}
one gets the following expression
$$(X(e_j, \eta'e_k) \cdot X(e_i, (\xi' \eta' + \pi(\xi)\eta' + \pi(\eta)\xi')e_k \cdot X(e_j, -\eta'e_k),\ x_{ik}(\pi(\xi\eta)).$$
Since $\xi'\eta' + \pi(\xi)\eta' + \pi(\eta)\xi' = \xi\eta - \pi(\xi\eta)$ the above expression clearly simplifies to $\psi(x_{ik}(\xi\eta))$, as claimed.
\end{proof}

\begin{rk} \label{rk:T123'} Notice that from the proof of the above proposition it follows that one could replace the family of relations \eqref{add3}
in the definition of $\St(n, R, I)$ with the following smaller family \eqref{add3'}.
\setcounter{equation}{2} \renewcommand{\theequation}{T\arabic{equation}'}
\begin{equation} X(e_ir+e_j,\,e_k)=X(e_i,\,rae_k)X(e_j,\,ae_k)\,\text{ for }r\in R,\ a\in I\text{ and } i,j,k\text{ distinct}. \label{add3'} \end{equation} \end{rk}

Both ``another presentations'' of van der Kallen and Tulenbaev are given in terms of generators parametrised by pairs of vectors,
 where the first one is ``good'' in some sense  (i.\,e. unimodular or a column of an elementary matrix) while the second one is arbitrary.
It is easy to formulate the additivity property in the second argument for these generators (cf. \eqref{add1}, \eqref{add2}),
while it is not so easy when it comes to the additivity in the first argument (cf. \eqref{add3}).
%Such presentations allowed them to formulate easily an additivity by the second argument, but one has to write something similar to T3 instead of additivity by the first one.

The construction of the homomorphism $T\colon\St(n,\,R_a[X],\,XR_a[X])\rightarrow\St(n,\,R\ltimes XR_a[X])$
amounts to choosing certain elements in the group $\St(n,\,R\ltimes XR_a[X])$ and proving that these elements satisfy relations \eqref{add2}--\eqref{add3}, see~\cite[Lemmas~1.2 and~1.3\,c)]{Tul}.
The main problem with this recipe is that the assumption $n\geq5$ is essential to check that the relation~\eqref{add3} holds.

It is not possible to choose arbitrary orthogonal vectors $u$, $v$ as parameters for the generators because in this situation it is not even known that $t(u,\,v)\in\E(n,\,R)$, see~\cite{Rao}. 

To generalise Tulenbaev's results for $n=4$ we will use a more ``symmetric`` presentation with two types of generators: 
$F(u,\,v)$ with $u$ ''good`` and $v$ arbitrary and $S(u,\,v)$ with $u$ arbitrary and $v$ ''good``.
The generators $F(u,\,v)$ will be additive in the second component, while $S(u,\,v)$ will be additive in the first one.
When $u$ and $v$ are both ''good`` we require these two generators to coincide.
More formally, we have the following presentation.

\begin{df}
For $I\trianglelefteq R$ and $n\geq4$ define $\St^*(n,\,R,\,I)$ to be the group with the set of generators
\begin{multline*}
\{F(u,\,v)\mid u\in\E(n,\,R)e_1,\ v\in I^n,\ u^tv=0\}\,\cup\{S(u,\,v)\mid u\in I^n,\ v\in\E(n,\,R)e_1,\ u^tv=0\}
\end{multline*}
subject to relations
\setcounter{equation}{0}
\renewcommand{\theequation}{R\arabic{equation}}
\begin{align}
&F(u,\,v)F(u,\,w)=F(u,\,v+w), \label{add4}\\
&S(u,\,v)S(w,\,v)=S(u+w,\,v), \label{add5}\\
&F(u,\,v)F(u',\,v')F(u,\,v)\inv=F(t(u,\,v)u',\,t(v,\,u)\inv v'), \label{conj3} \\
&F(u,\,va)=S(ua,\,v),\ \text{for all}\ a\in I,\,(u,\,v^t)=(M e_1,\,e_2^t M\inv),\,M\!\in\E(n,\,R) \label{coinc}.
\end{align}
\end{df}
Notice that we only described how a generator $F(u, v)$ acts via conjugation on another generator $F(u',v')$ and have omitted three similar relations involving generators $S(u,v)$ of the second type.
The reason for this is the following lemma which asserts that the ``missing'' relations will follow automatically from \eqref{add4}--\eqref{coinc}.
\begin{lm}
\label{allyouneedisf}
Denote by $\phi\colon\St^*(n,\,R,\,I)\rightarrow\E(n,\,R)$ the natural map sending $F(u,\,v)\mapsto t(u,\,v)$ and $S(u,\,v)\mapsto t(u,\,v)$.
Then the following facts are true.
\begin{lmlist}
\item $\St^*(n,\,R,\,I)$ is generated as an abstract group by the set of elements 
      $$\{F(u,\,va)\mid a\in I,\ (u,\,v^t)=(Me_1,\,e_2^tM\inv),\,M\in\E(n,\,R)\};$$
\item for any $g\in\St^*(n,\,R,\,I)$ one has 
      \setcounter{equation}{2} \renewcommand{\theequation}{R\arabic{equation}'}
      \begin{equation} gF(u,\,v)g\inv=F(\phi(g)u,\,\phi(g\inv)^tv); \end{equation}
\item for any $g\in\St^*(n,\,R,\,I)$ one has
       \setcounter{equation}{2} \renewcommand{\theequation}{R\arabic{equation}''}
      \begin{equation} gS(u,\,v)g\inv=S(\phi(g)u,\,\phi(g\inv)^tv); \end{equation}
\item there is a ``transpose automorphism'' defined on $\St^*(n, R, I)$ satisfying
      $$F(u,\,v)^t=S(v,\,u),\quad S(u,\,v)^t=F(v,\,u).$$
\end{lmlist} \end{lm}

\begin{proof}
Let $F(u,\,v)$ be an arbitrary generator of the first type and let $M\in\E(n,\,R)$ be such that $F(u,\,v)=F(Me_1,\,M^*\tilde v)$. 
Here we denote by $M^*$ the contragradient matrix, $M^*=(M\inv)^t$. 
By \eqref{add4} we have $$F(u,\,v)=\prod\limits_{k\neq1}F(Me_1,\,M^*e_k\tilde v_k)$$
where $\tilde v_k$ stands for $k$-th coordinate of $\tilde v=\sum e_i\tilde v_i$.
Applying relations \eqref{add5}, \eqref{coinc} we get
$$S(u,\,v)=\prod\limits_{k\neq1}S(Ne_k\tilde u_k,\,N^*e_1)=\prod\limits_{k\neq1}F(Ne_k,\,N^*e_1\tilde u_k)$$
and, thus, obtain $(\mathrm{i})$. Obviously, $(\mathrm{ii})$ follows from $(\mathrm{i})$. 
To prove $(\mathrm{iii})$ it suffices to show that
$$F(u,\,v)S(u',\,v')F(u,\,v)\inv=S(t(u,\,v)u',\,t(u,\,v)^*v').$$
For $S(u',\,v')=\prod_{k\neq 1} F(Ne_k,\,N^*e_1\tilde u_k)$ we have
\begin{multline*}
F(u,\,v)S(u',\,v')F(u,\,v)\inv=F(u,\,v)\prod F(Ne_k,\,N^*e_1\tilde u_k)F(u,\,v)\inv=\\
=\prod F(t(u,\,v)Ne_k,\,t(u,\,v)^*N^*e_1\tilde u_k)=\\
=\prod S(t(u,\,v)Ne_k\tilde u_k,\,t(u,\,v)^*N^*e_1)=S(t(u,\,v)u',\,t(u,\,v)^*v').
\end{multline*}
Finally, $(\mathrm{iv})$ follows from $(\mathrm{iii})$.
\end{proof}

Our next goal is to show for a splitting ideal $I\trianglelefteq R$ that the group $\St^*(n,\,R,\,I)$ is isomorphic to $\St(n,\,R,\,I)=\Ker\big(\St(n,\,R)\epi\St(n,\,R/I)\big)$.
With this end we construct two mutually inverse homomorphisms
$$\xymatrix{\St^*(n,\,R,\,I)\ar@<0.5ex>[r]^{\iota}&\St(n,\,R,\,I)\ar@<0.5ex>[l]^{\kappa}.}$$

In order to construct the arrow $\kappa$ we use the presentation from \cref{prop:TulPres}.
We set $$\kappa(X(u,v)) = F(u,v),\ u\in \E(n, R)e_1,\ v\in I^n.$$ 
In view of \cref{rk:T123'} it suffices to check that $F(u,v)$ satisfy the relations \eqref{add2}, \eqref{conj2}, \eqref{add3'}.
The validity relations \eqref{add2}, \eqref{conj2} is obvious while \eqref{add3'} follows from the following computation:
$$ F(e_i, rae_k) F(e_j, ae_k) = S(rae_i, e_k) S(ae_j, e_k) = S(a(re_i+e_j), e_k) = F(e_ir+e_j, ae_k). \qedhere $$

As for the map $\iota$ we firstly define it as a homomorphism to the absolute group (for which we de not need the assumption that $I$ is splitting).
$$\xymatrix{\St^*(n,\,R,\,I)\ar@<0.0ex>[r]^{\iota}&\St(n,\,R).}$$

Clearly, we must map the elements $F(u,\,v)$ to van der Kallen's generators $X(u,\,v)$.
We should find the images for the elements $S(u,\,v)$ as well.
First of all, recall that van der Kallen in~\cite[3.8--3.10]{vdK} constructs the following elements
$$x(u,\,v)\in\St(n,\,R),\ u^tv=0,\ u_i=0\,\text{ or }\,v_i=0\,\text{ for some }\,1\leq i\leq n.$$
The following definition is the ``transpose'' of \cite[3.13]{vdK}.
\begin{df} For $u\in R^n$, $v\in\E(n,\,R)e_1$, $u^tv=0$, consider the set $\overline Y(u,\,v)\subseteq\St(n,\,R)$
 consisiting of all elements $y\in\St(n,\,R)$ that admit the decomposition $y=\prod x(w^k,\,v)$, 
 where $\sum u^k=u$ and $u^k=c^k \cdot (e_pv_q-e_qv_p)$ for some $c^k\in R$, $1\leq p\neq q\leq n$. \end{df}

Since columns of elementary matrices are unimodular, one can find $w\in R^t$ such that $w^tv=1$.
It is not hard to check that $u = (w^t\cdot v)\cdot u = \sum_{p<q}u_{pq}$, where $u_{pq} = (u_pw_q - u_qw_p) \cdot (v_qe_p - v_pe_q)\in{}\!R^n$,
therefore $\overline Y(u,\,v)$ is not empty (cf.~\cite[3.1--3.2]{vdK}).
Obviously, for $x\in\overline Y(u,\,v)$ and $y\in\overline Y(w,\,v)$ one has $xy\in\overline Y(u+w,\,v)$.

Repeating~\cite[3.14--3.15]{vdK} verbatim one shows the following.
\begin{lm} For $g\in\St(n,\,R)$ one has
\begin{enumerate} \item $g\overline Y(u,\,v)g\inv\subseteq\overline Y(\phi(g)u,\,\phi(g)^*v)$;
                  \item $\overline Y(u,\,v)$ consists of exactly one element. \end{enumerate} \end{lm}

We denote the only element of $\overline Y(u,\,v)$ by $Y(u,\,v)$. 
Now we can finish the definition of $\iota$ by requiring that $\iota(S(u,\,v)) = Y(u,\,v)$ for $S(u,\,v)\in\St^*(n,\,R,\,I)$. 
To show that the map $\iota$ is well-defined we should check that elements $X(u,\,v)$ and $Y(u,\,v)$ satisfy the relations 
\eqref{add4}--\eqref{coinc} (with $F$'s and $S$'s replaced with $X$'s and $Y$'s).
Only the validity of relation \eqref{coinc} is not immediately obvious.
\begin{lm} \label{lm:XY} For $(u,\,v^t)=(Me_1,\,e_2^tM\inv)$, $M\in\E(n,\,R)$, $a\in I$ one has $X(u,\,va)=Y(ua,\,v)$. \end{lm}
\begin{proof}
In view of the above lemma we only need to show that $X(e_1,\,e_2a)=Y(e_1a,\,e_2)$.
This equality can be obtained by computing the commutator $[Y(-e_3,\,e_2),\,X(e_1,\,e_3a)]$ in two ways:
$$ Y(-e_3,\,e_2)X(e_1,\,e_3a)Y(-e_3,\,e_2)\inv\cdot X(e_1,\,-e_3a)=X(e_1,\,e_2a), \text{ and} $$
$$ Y(-e_3,\,e_2)\cdot\,X(e_1,\,e_3a)Y(e_3,\,e_2)X(e_1,\,e_3a)\inv=Y(e_1a,\,e_2). \qedhere $$
\end{proof}

The following result is a consequence of the above argument.
\begin{lm} \label{lm:map-iota}
 For any $I \trianglelefteq R$ there exists a well-defined homomorphism $$\iota\colon\St^*(n,\,R,\,I)\rightarrow\St(n,\,R, I) \subseteq \St(n, R)$$ 
 satisfying the equalities $\iota (F(u,\,v)) = X(u,\,v)$ and $\iota (S(u,\,v)) = Y(u,\,v)$. \end{lm}
\begin{proof}
We already know that $\iota$ is well-defined. It suffices to check that the image of $\iota$ is contained in $\St(n,\,R,I)$.
Denote the canonical projection $R\epi R/I$ by $\rho$.
Clearly, the induced group morphism $\rho^*\colon\St(n,\,R)\rightarrow\St(n,\,R/I)$ sends $X(u,\,v)$ to $X(\rho(u),\,\rho(v))$ and $Y(u,\,v)$ to $Y(\rho(u),\,\rho(v))$. 
Observe that $\E(n,\,R)\epi\E(n,\,R/I)$ is surjective, therefore for $x\in\E(n,\,R)e_1$ one has $\rho(x)\in\E(n,\,R/I)e_1$. 
From the additivity relations we get that $X(u,\,0)=1=Y(0,\,v)$, hence $\mathrm{Im}(\iota)\subseteq\Ker(\rho^*)=\St(n,\,R,\,I)$, as required. \end{proof}
 
\begin{comment}
We constructed $\iota$ for an arbitrary $I\trianglelefteq R$. For $I$ a splitting ideal we construct an inverse homomorphism
$$
\kappa\colon\St(n,\,R,\,I)\rightarrow\St^*(n,\,R,\,I).
$$
With this end we recall a presentation of $\St(n,\,R,\,I)$ as a group with action of $\St(n,\,R)$. For a splitting ideal such a presentation was firstly obtained by Swan~\cite{Swa1,Swa2,Swa3}, and for an arbitrary $I$ by Keune~\cite{Keu} and Loday~\cite{Lod}.

For a group $G$ acting on a group $H$ from the left denote by $\!\,^gh$ the image of $h\in H$ under homomorphism corresponding by $g\in G$.

Let $I\trianglelefteq R$ be a splitting ideal then $\St(n,\,R,\,I)$ can be presented as a group with action of $\St(n,\,R)$ with the set of (relative) generators
$$
\{y_{ij}(a)\mid a\in I,\ 1\leq i\neq j\leq n\}
$$
subject to relations
\setcounter{equation}{0}
\renewcommand{\theequation}{KL\arabic{equation}}
\begin{align}
&y_{ij}(a)y_{ij}(b)=y_{ij}(a+b),\\
&\!\,^{x_{ij}(r)}y_{hk}(b)=y_{hk}(b),\text{ for }h\neq j,\ k\neq i,\\
&\!\,^{x_{ij}(r)}y_{jk}(b)\cdot y_{jk}(-b)=y_{ik}(rb),\\
&y_{ij}(a)\cdot\,^{x_{jk}(s)}y_{ij}(-a)=y_{ik}(as),\\
&\!\,^{x_{hk}(a)}\big(\!\,^gy_{ij}(b)\big)=y_{hk}(a)\cdot\,^gy_{ij}(b)\cdot y_{hk}(-a)\,\text{ for }a\in I,\ g\in\St(n,\,R).
\end{align}

In other words, $\St(n,\,R,\,I)$ is isomorphic to a quotient group of a free group $F$ on the set of generators 
$$\St(n,\,R)\times\{y_{ij}(a)\mid a\in I,\ 1\leq i\neq j\leq n\}$$
(the natural action of $\St(n,\,R)$ on $F$ is given by $\!\,^f(g,\,y)=(fg,\,y)$) modulo the normal equivariant subgroup generated by KL1--KL5 (where $y_{ij}(a)$ stands for $(1,\,y_{ij}(a))$. Keune in Loday consider only stable Steinberg group $\St(R)=\varinjlim\St(n,\,R)$. This presentation for unstable group firstly appeared in~\cite{SCh} in the context of Chevalley groups. Observe that the definition of the relative Steinberg group in~\cite[Def.~3.3]{SCh} differs from ours but remark after this definition and~\cite[Lem.~8]{SCh} immediantly imply that for a splitting ideal $I\trianglelefteq R$ it coincides with the one given in the present paper. The existance of the presentation is proven in~\cite[Prop.~6]{SCh}. Relation~5 in~\cite[Prop.~6]{SCh} joins relations KL3 and KL4, and relations~2 and 3 follow from KL4 and KL5 with the use of KL6.

One could probably wonder why Tulenbaev needs his own presentation for $\St(n,\,R,\,I)$ and does not use the one of Keune--Loday. The reason is the following. To construct a map from Keune--Loday group one should firstly define an action of the absolute Steinberg group on the target group, in our context, the action of $\St(n,\,B_a)$ on $\St(n,\,B,\,I)$, what is probably possible, but seems to be much harder, then giving another presentation (in particular, one should define $\!\,^{x_{ij}(r/a^m)}y_{ji}(c)$).

Now, to construct
$
\kappa\colon\St(n,\,R,\,I)\rightarrow\St^*(n,\,R,\,I)
$
we need to define an action of $\St(n,\,R)$ on $\St^*(n,\,R,\,I)$ and find elements $y^*_{ij}(a)\in\St^*(n,\,R,\,I)$ subject to KL1--KL5. To define a action of the absolute group we use van der Kallen's another presentaion for it. For $u\in\Um(n,\,R)$, $v\in R^n$, $u^tv=0$ define
$
\alpha(u,\,v)\colon\St^*(n,\,R,\,I)\rightarrow\St^*(n,\,R,\,I)
$
by $\alpha(u,\,v)\big(F(u',\,v')\big)=F(t(u,\,v)u',\,t(u,\,v)^*v')$, and $\alpha(u,\,v)\big(S(u',\,v')\big)=S(t(u,\,v)u',\,t(u,\,v)^*v')$. Obviously, the images of the generators satisfy R1--R4, so that $\alpha(u,\,v)$ is a well-defined automorphism. Also, $\alpha$'s themselves clearly satisfy K1--K2, thus $X(u,\,v)\mapsto\alpha(u,\,v)$ is a well-defined action of $\St(n,\,R)$ on $\St^*(n,\,R,\,I)$.

Next, define $y_{ij}^*(a)=F(e_i,\,e_ja)$, $a\in I$. These elements obviously satisfy KL1, KL2 and KL5. Check KL3:
\begin{multline*}
\!\,^{x_{ij}(r)}y^*_{jk}(b)=F(t_{ij}(r)e_j,\,t_{ji}(-r)e_kb)=F(e_ir+e_j,\,e_kb)=\\
=S(e_irb+e_jb,\,e_k)=S(e_irb,\,e_k)S(e_jb,\,e_k)=F(e_i,\,e_krb)F(e_j,\,e_kb).
\end{multline*} 
KL4 is similar. Finally, we have a well-defined map
$$
\kappa\colon\St(n,\,R,\,I)\rightarrow\St^*(n,\,R,\,I)
$$
sending $y_{ij}(a)$ to $y^*_{ij}(a)$. Obviously, $\iota\circ\kappa=\mathrm{id}$ (cf.~\cite[3.6\,d)]{vdK}), thus $\kappa$ is injective. Surjectivity of $\kappa$ follows from Lemma~\ref{allyouneedisf}\,a), thus it is an isomorphism and $\iota$ is inverse to it.
\end{comment}

At this point we have demonstrated for $n\geq4$ and a splitting ideal $I$ that $\St^*(n,\,R,\,I)$ and $\St(n,\,R,\,I)$ are isomorphic.
Now we turn to the main result of this section, namely, we construct a map $$T\colon\St(n,\,R_a[X],\,XR_a[X])\rightarrow\St(n,\,R\ltimes XR_a[X],\,XR_a[X]),$$ for $n\geq4$. 
A local-global principle for Steinberg group and centrality of $\mathrm K_2$ formally follow from existance of this map~\cite[Lem.~15, Lem.~16, proof of Th.~2]{SCh}.

We work in a more general situation.

\begin{tm}
\label{a3map}
Let $B$ be a ring, $I\trianglelefteq B$, $a\in B$ such that $\forall\,x\in I$ there exists a unique $y\in I$ such that $ya=x$ {\rm(}we denote this $y$ by $\frac xa$, elements $\frac x{a^m}$ are also well defined{\rm)}. This requirement is equivalent to say that the principle localisation $\lambda_a\colon I\rightarrow I_a$ of ideal $I$ is an isomorphism. Then there exists a map
$$
\mathrm T\colon\St(n,\,B_a,\,I)\rightarrow\St(n,\,B)
$$
making the diagram
$$
\xymatrix{
\St^*(2n,\,B,\,I)\ar@<-0.0ex>[rr]^{\iota}\ar@<-0.0ex>[d]_{\lambda_{a}^*}&&\St(2n,\,B)\ar@<-0.0ex>[d]^{\lambda_{a}^*}\\
\St^*(2n,\,B_a,\,I)\ar@<-0.0ex>[rr]_{\iota}\ar@{-->}[rru]_{\mathrm T}&&\St(2n,\,B_a)
}
$$
commutative.
\end{tm}

To prove this theorem we need to find some elements ``$X(u,\,v)$'' inside $\St(n,\,B)$ for $u\in\E(n,\,B_a)e_1$, $v\in I$. The idea is the following. Say $a$ is not a zero divisor, then $ua^m\in B$ for some $m$ and we will construct some element ``$X(ua^m,\,va^{-m})$''. One can not garantee that $ua^m\in\E(n,\,B)e_1a^k$, however, the ideal generated by entries of $ua^m$ contains some power of $a$ (this is equivalent to say that $ua^m$ becomes unimodular after the principle localisation in $a$). Denote $\mathrm I(u)$ the ideal generated by entries of $u\in R^n$, $\mathrm I(u)=\sum\limits_{k=1}^nu_kR$. Then, we need generators parametrised by pairs $(u,\,v)$ with $u^tv=0$, $v\in I$, $a^m\in\mathrm I(u)$ for some $m\in\mathbb N$. Such generators are defined in~\cite{Tul}.

\begin{df}[Tulenbaev]
For $u$, $v\in B^n$, $a\in\mathrm I(u)$ and $v_1,\ldots,v_N\in B^n$ such that $u^tv_k=0$ $\forall\,k$,  each $v_k$ has at least two zero coordinates and $\sum_{k=1}^Nv_k=v$ define $X_{u,v}(a)=\prod\limits_{k=1}^Nx(u,\,v_ka)$. Tulenbaev shows that factors $x(u,\,v_ka)$ commute~\cite[Lem.~1.1\,e)]{Tul} and that for any other decomposition of $v$ as a sum $v=\sum_{j=1}^Mv'_j$ of vectors orthogonal to $u$ and having two zero coordinates holds $\prod_{k=1}^Nx(u,\,v_ka)=\prod_{j=1}^Mx(u,\,v'_ja)$~\cite[p.~3]{Tul}, i.e., elements $X_{u,v}(a)$ are well-defined. Obviously, $\phi(X_{u,v}(a))=t(u,\,va)$.
\end{df}

Observe that Tulenbaev uses different notation for van der Kallen elements. He writes $X_{u,v}$ instead of $X(u,\,v)$ and $X(u,\,v)$ instead of $x(u,\,v)$. We keep van der Kallen's notation.

\begin{lm}
\label{xproperties}
For $u$, $v$, $v'$ and $w\in B^n$, such that $v$ and $v'$ have decomposition as in above definition, $u^tw=0$, $a$, $b\in\mathrm I(u)$, $c\in B$, $g\in\St(n,\,B)$ holds
\begin{enumerate}
\item
$X_{u,vc}(a)=X_{u,v}(ca)$,
\item
$X_{uc,v}(ca)=X_{u,vc^2}(a)$,
\item
$X_{u,v}(a)X_{u,v'}(a)=X_{u,v+v'}(a)$,
\item
$g\,X_{u,wb}(a)g\inv=X_{\phi(g)u,\phi(g)^*wb}(a)$.
\end{enumerate}
\end{lm}

\begin{proof}
The statement of a) is obvious from the definition, b) follows from~\cite[Lem.~1.1\,d)]{Tul}, c) is~\cite[Lem.~1.3\,a)]{Tul}, d) is proven for $n\geq5$ in~\cite[Lem.~1.3\,b)]{Tul} and afterwards Tulenbaev makes a remark that for $n=4$ the statement is also true. Indeed, take $z\in B^n$ such that $z^tu=b$ and decompose
$$
(z^tu)w=\sum_{i<j}w_{ij},
$$
where $w_{ij}=(e_iu_j-e_ju_i)(w_iz_j-w_jz_i)=w_{ji}$. Each $w_{ij}$ is orthogonal to $u$ and has two zero coordinates ($n\geq4$). Thus, $X_{u,wb}(a)=\prod_{i<j}x(u,\,w_{ij}a)$. It is enough to prove the statement of d) for $g=x_{hk}(r)$. If $h\neq i,j$ or $\{h,k\}=\{i,j\}$, $\phi(g)^*w_{ij}a$ still has two zero coordinates. Consider the case $j=h$, $i\neq k$. With~\cite[3.12]{vdK} we get $g\,x(u,\,w_{ij}a)g\inv=x(\phi(g)u,\,\phi(g)^*w_{ij}a)$. Denote $u_{ij}=e_iu_j-e_ju_i$ and $c_{ij}=w_iz_j-w_jz_i$. Using
$$
\phi(g)^*u_{ij}=(\phi(g)u)_{ij}+(\phi(g)u)_{ki}\,r
$$
and~\cite[3.11]{vdK} one obtains
$$
x(\phi(g)u,\,\phi(g)^*w_{ij}a)=x(\phi(g)u,\,(\phi(g)u)_{ij}c_{ij}a)\cdot x(\phi(g)u,\,(\phi(g)u)_{ki}\,rc_{ij}a).
$$
Decomposing in such a way each $\phi(g)^*w_{ij}a$ which does not have two zero coordinates we finally get a product from the definition of $X_{\phi(g)u,\phi(g)^*wb}(a)$.
\end{proof}

Siilarly, one can define ``transposed'' version of Tulenbaev's elements $X_{u,v}(a)$.

\begin{df}
For $u$, $v\in B^n$, $a\in\mathrm I(v)$, and $u_1,\ldots,u_N\in B^n$ such that $u_k^tv=0$, each $u_k$ has at least two zero coordinates and $u=\sum_{k=1}^Nu_k$ define $Y_{u,v}(a)=\prod\limits_{k=1}^Nx(u_ka,\,v)$. Repeating Tulenbaev's argumentation~\cite[p.~3]{Tul} one can show that the definition does not depend on the order of factors and on choice of decomposition of $u$.
\end{df}

Now, one can repeat van der Kallen's and Tulenbaev's argumentation to prove transposed version of Lemma~\ref{xproperties}. We leave this to the reader.

\begin{lm}
\label{yproperties}
For $u$, $u'$, $w$ and $v\in B^n$, such that $u$ and $u'$ have decomposition as in above definition, $w^tv=0$, $a$, $b\in\mathrm I(v)$, $c\in B$, $g\in\St(n,\,B)$ holds
\begin{enumerate}
\item
$Y_{uc,v}(a)=Y_{u,v}(ca)$,
\item
$Y_{u,vc}(ca)=Y_{uc^2,v}(a)$,
\item
$Y_{u,v}(a)X_{u',v}(a)=Y_{u+u',v}(a)$,
\item
$g\,Y_{wb,v}(a)g\inv=Y_{\phi(g)wb,\phi(g)^*v}(a)$.
\end{enumerate}
\end{lm}

Finally, it only remains to show that for a ``good'' pair $(u,\,v)$ elements $X$ and $Y$ coincide.

\begin{lm}
\label{x=y}
Consider $w$, $u$, $z$, $v$, $x$, $y\in B^n$ $a$, $r\in B$ such that $w^tu=a$, $z^tv=a$, $x^ty=a$ and pairs $(w,\,u)$, $(z,\,v)$ and $(x,\,y)$ are mutually orthogonal. Then 
$$
X_{u,v}(ra^3)=Y_{u,v}(ra^3).
$$
\end{lm}

\begin{proof}
On one hand,
\begin{multline*}
[Y_{x,-v}(ra),\,X_{u,y}(a)]=X_{t(x,-vra)u,t(x,-vra)^*y}(a)X_{u,-y}(a)=\\
=X_{u,y+vra^2}(a)X_{u,-y}(a)=X_{u,v}(ra^3).
\end{multline*}
On the other,
\begin{multline*}
[Y_{x,-v}(ra),\,X_{u,y}(a)]=Y_{x,-v}(ra)Y_{t(u,ya)x,t(u,ya)^*v}(ra)=\\
=Y_{-x,v}(ra)Y_{x+ua^2,v}(ra)=Y_{u,v}(ra^3).
\end{multline*}
\end{proof}

Now, we are ready to construct a map 
$
\mathrm T\colon\St(n,\,B_a,\,I)\rightarrow\St(n,\,B).
$

\begin{proof}[Proof of Theorem~\ref{a3map}]
For $u=Me_1$, $M\in\E(n,\,B_a)$, $v\in I^n$, $u^tv=0$ denote $w=M^*e_1$ (then $wu=1$) and choose $m\in\mathbb N$ such that some $\tilde w$, $\tilde u\in B^n$ localise to $wa^m$ and $ua^m$, $\tilde u^tv=0$ and $\tilde w^t\tilde u=a^{2m}$. Now set $\mathrm T(F(u,\,v))=X_{\tilde u,v/a^{3m}}(a^{2m})$. Lemma~\ref{xproperties}\,a) and b) garantees that this definition does not depend on the choice of $m$ and the lifts $\tilde u$ and $\tilde w$. Similarly, set $\mathrm T(S(u,\,v))=Y_{u/a^{3m},\tilde v}(a^{2m})$. These elements satisfy R1--R3 by Lemmas~\ref{xproperties} and~\ref{yproperties}. For $u=Me_1$, $v=M^*e_2$, $M\in\E(n,\,B_a)$ consider $w=M^*e_1$, $z=Me_2$, $x=Me_3$ and $y=M^*e_3$. Then, multiply these element on appropriate power of $a$, take their lifts to $B$ and apply Lemma~\ref{x=y} to get R4. So that, the map $\mathrm T$ is well-defined.

Commutativity of the diagram follows directly from the definitions of elements $X(u,\,v)$, $Y(u,\,v)$ and $X_{u,v}(a)$, $Y_{u,v}(a)$ and the possibility to redistribute the powers of $a$ after localisation.
\end{proof}

\section{Patching Tulenbaev maps}\label{sec:patching}

The next section is devoted to the proof of Theorem~\ref{a3map} for Steinberg groups corresponding to an arbitrary simply-laced root system of rank $l\geq3$. To do so, we will ``glue'' maps $T$ constructed in the present section for systems of type $\mathrm A_3$. All main results of the present paper follow from the existence of such a map.

\printbibliography

\end{document}

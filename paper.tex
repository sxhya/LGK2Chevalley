\documentclass[12pt]{amsart}
\usepackage{amscd, amsmath, amssymb, amsthm, amsfonts, amstext, verbatim, enumitem, graphicx, mathtools, xfrac, tikz-cd, nameref, thmtools, hyperref}
\usepackage[utf8]{inputenc}
\usepackage[backend=biber, citestyle=numeric-comp, natbib=true, sortlocale=en_US, url=false, doi=false, eprint=true, maxbibnames=4]{biblatex}            
\addbibresource{paper.bib}
\usepackage[activate={true,nocompatibility}]{microtype}
\usepackage[capitalize]{cleveref}
            
\newlist{thmlist}{enumerate}{1} \setlist[thmlist]{label=(\roman{thmlisti}), ref=\thethm.(\roman{thmlisti}),noitemsep} \Crefname{thmlisti}{Theorem}{Theorems}
\newlist{lemlist}{enumerate}{1} \setlist[lemlist]{label=(\roman{lemlisti}), ref=\thelemma.(\roman{lemlisti}),noitemsep} \Crefname{lemlisti}{Lemma}{Lemmas}

\theoremstyle{plain} \declaretheorem[name=Theorem, Refname={Theorem,Theorems}]{thm} \Crefname{thm}{Theorem}{Theorems}
\numberwithin{equation}{section}

\newtheorem{lemma}{Lemma} \numberwithin{lemma}{section} \Crefname{lemma}{Lemma}{Lemmas}
\newtheorem{cor}[lemma]{Corollary} \Crefname{cor}{Corollary}{Corollaries}
\newtheorem{prop}[lemma]{Proposition} \Crefname{prop}{Proposition}{Propositions}
\newtheorem*{thm*}{Theorem}
\newtheorem*{lemma*}{Lemma}
\theoremstyle{definition} \newtheorem{dfn}[lemma]{Definition} \Crefname{dfn}{Definition}{Definitions}
\newtheorem{example}[lemma]{Example} \Crefname{example}{Example}{Examples}
\theoremstyle{remark} \newtheorem{rem}[lemma]{Remark} \Crefname{rem}{Remark}{Remarks}

%Page layout
\textwidth 16cm 
\textheight 22cm 
\headheight 0.5cm 
\evensidemargin 0.3cm 
\oddsidemargin 0.2cm

\newcommand{\catname}[1]{{\normalfont\textbf{#1}}}
\newcommand{\Ker}{\operatorname{Ker}}
\newcommand{\Cent}{\operatorname{Cent}}
\newcommand{\Img}{\operatorname{Im}}
\newcommand{\Coker}{\operatorname{Coker}}
\newcommand{\coker}{\operatorname{coker}}
\newcommand{\K}{\operatorname{\mathrm{K}}}
\newcommand{\GG}{\operatorname{\mathrm{G}}}
\newcommand{\St}{\operatorname{\mathrm{St}}}
\newcommand{\StU}{\operatorname{\hat{\mathrm{U}}}}
\newcommand{\G}{\operatorname{\mathrm{G}}}
\newcommand{\E}{\operatorname{\mathrm{E}}}
\newcommand{\M}{\operatorname{\mathrm{M}}}
\newcommand{\sr}{\operatorname{\mathrm{sr}}}
\newcommand{\Um}{\operatorname{\mathrm{Um}}}
\newcommand{\rk}{\operatorname{\mathrm{rk}}}
\newcommand{\rA}{\mathsf{A}}
\newcommand{\rB}{\mathsf{B}}
\newcommand{\rC}{\mathsf{C}} 
\newcommand{\rD}{\mathsf{D}} 
\newcommand{\rE}{\mathsf{E}}
\newcommand{\HH}{\mathrm{H}}
\newcommand{\eval}[4]{ev \textstyle \left[\frac{#2[#1] \rightarrow #3}{#1 \mapsto #4}\right]}
\newcommand{\ev}[3]{\eval{t}{#1}{#2}{#3}}

\newcommand\restr[2]{{\left.\kern-\nulldelimiterspace #1 \vphantom{\big|}\right|_{#2}}}
\newcommand*\MapsTo{\xrightarrow[\raisebox{0.25 em}{\smash{\ensuremath{\sim}}}]{}}  

\title{Local-global principle for $\K_2$ of Chevalley groups}
\keywords {Chevalley groups, Steinberg groups, $K_2$-functor. {\em Mathematical Subject Classification (2010):} 19C09; 20G35}
\author {Andrei Lavrenov, Sergey Sinchuk}
\email {avlavrenov {\it at} gmail.com, sinchukss {\it at} gmail.com}
\date {\today}

\begin{document}

\begin{abstract} To be written. \end{abstract}

\maketitle

\section {Introduction}\label{intro}
 
\subsection{Acknowledgements}
The authors of the present paper acknowledges financial support from Russian Science Foundation grant 14-11-00297.

\section {Preliminaries}\label{prelim}

\section {Main results}

\subsection{Relative Steinberg group in rank $3$ and Tulenbaev's section map} 
The main goal of this subsection is to construct Tulenbaev's section map from~\cite[\S~2]{T} for
the linear Steinberg group of rank $3$. Combining this result with the amalgamation theorem
of the the second named author (see~\cite[Theorem~9]{S15}) we deduce centrality for the orthogonal
$\K_2$ in the next subsection.

First of all, we will need to give yet another definition of the relative linear Steinberg group 
(cf. \cite[Definitions~3.3 and 3.7]{S15}).

\begin{dfn}
 The relative Steinberg group $\St^*(n,R, I)$ is the group defined by the following two
 families generators and four families of relations.
 \begin{itemize}
  \item Generators:
  \begin{enumerate}
  \item $X^1(u, v)$, where $u \in \E(n,R) \cdot e_1$, $v\ \in I^n$ such that $v^t \cdot u = 0$;
  \item $X^2(u, v)$, where $u \in I^n$, $v \in \E(n,R) \cdot e_1$ such that $v^t \cdot u = 0$.
 \end{enumerate}
  Notice that the canonical image in $\E(n, R)$ of both $X^1(u, v)$ and $X^2(u, v)$ equals $T(u, v) = e_n + u \cdot v^T$.
  \item Relations:
  \begin{enumerate}
  \item $X^1(u, v) \cdot X^1(u, w) = X^1(u, v+w)$, $u \in \E(n,R) \cdot e_1$, $v, w \in I^n$;
  \item $X^2(u, v) \cdot X^2(w, v) = X^2(u+w, v)$, $u, w \in I^n$, $v \in \E(n,R) \cdot e_1$;
  \item ${}^{X^\sigma(u^2, v^2)} \! X^\tau(u^1, v^1) = X^\tau(T(u^2, v^2) \cdot u^1, T(v^2, u^2)^{-1} \cdot v^1)$, $\sigma, \tau = 1..2$;
  \item $X^1(u, vb) = X^2(ub, v)$ for $(u,v) \in E(n,R) \cdot (e_1, e_2)$, $b\in I$.
 \end{enumerate}
 \end{itemize}
\end{dfn}
It is easy to check correctness of the above definition.

\begin{lemma}
 The groups $\St^*(n, R, I)$ and $\St(n, R, I)$ are isomorphic.
\end{lemma}
\begin{proof}
 TODO: 
\end{proof}

The next step is construct 


\begin{dfn}
 Let $u$,$v$ be elements of $B^n$ such that $v^Tu = 0$ and there exists $w \in B^n$ such that $<w,u> = a^N$.
 In this context we define elements $Z^1(u,v)$ and 
\end{dfn}

for each $(u,v) \in \E_n(R) (e_1, e_2)$ we can find $y$ such that $(u,v,y) = g (e_1, e_2, e_3)$.

$w^tu = a^N$, $z^Tv = a^N$, $x^T y = a^N$ + 6 additional orthogonality relations

Computing $g = [Z^{1, N}(u xba^{2N}), Z^{2, N}(ua^{2N},v)]$ 
$$ g = Z^{2, N}(ya^{2N} + uba^{5N},v) {Z^{2, N}(ya^{2N}, v)}^{-1} = Z^{2, N}(uba^{5N},v) = Z^2(uc, v)$$
$$ g = ... = Z^1(u, vc)$$

\subsection{Centrality of orthogonal $K_2$.} 

\printbibliography

\end{document}

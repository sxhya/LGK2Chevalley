\documentclass[11pt]{amsart}
\usepackage{amscd, amsmath, amssymb, amsthm, amsfonts, amstext, verbatim, enumitem, mathtools, xfrac, nameref, thmtools, hyperref}
\usepackage[utf8]{inputenc}
\usepackage[backend=biber, citestyle=numeric-comp, natbib=true, sortlocale=en_US, url=false, doi=false, eprint=true, maxbibnames=4]{biblatex}            
\usepackage[activate={true,nocompatibility}]{microtype}
\usepackage[capitalize]{cleveref}
\usepackage[english]{babel} 
\usepackage{stmaryrd}
\usepackage[matrix,arrow,curve]{xy}

%Bibliography
\renewbibmacro{in:}{\ifentrytype{article}{}{\printtext{\bibstring{in}\intitlepunct}}}
\newbibmacro{string+doi}[1]{\iffieldundef{doi}{\iffieldundef{url}{#1}{\href{\thefield{url}}{#1}}}{\href{http://dx.doi.org/\thefield{doi}}{#1}}}
\DeclareFieldFormat{title}{\usebibmacro{string+doi}{\mkbibemph{#1}}}
\DeclareFieldFormat[article, inproceedings, inbook, thesis]{title}{\usebibmacro{string+doi}{\mkbibquote{#1}}}
\addbibresource{paper.bib}
            
\newlist{thmlist}{enumerate}{1} \setlist[thmlist]{label=(\roman{thmlisti}), ref=\thethm.(\roman{thmlisti}),noitemsep} \Crefname{thmlisti}{Theorem}{Theorems}
\newlist{lemlist}{enumerate}{1} \setlist[lemlist]{label=(\roman{lemlisti}), ref=\thelemma.(\roman{lemlisti}),noitemsep} \Crefname{lemlisti}{Lemma}{Lemmas}

\theoremstyle{plain} \declaretheorem[name=Theorem, Refname={Theorem,Theorems}]{tm} \Crefname{tm}{Theorem}{Theorems}
\numberwithin{equation}{section}

\newtheorem{lm}{Lemma} \numberwithin{lm}{section} \Crefname{lm}{Lemma}{Lemmas}
\newtheorem{cl}[lm]{Corollary} \Crefname{cl}{Corollary}{Corollaries}
\newtheorem{prop}[lm]{Proposition} \Crefname{prop}{Proposition}{Propositions}
\newtheorem*{tm*}{Theorem}
\newtheorem*{lm*}{Lemma}
\theoremstyle{definition} \newtheorem{df}[lm]{Definition} \Crefname{df}{Definition}{Definitions}
\newtheorem{example}[lm]{Example} \Crefname{example}{Example}{Examples}
\theoremstyle{remark} \newtheorem{rk}[lm]{Remark} \Crefname{rk}{Remark}{Remarks}
\newtheorem*{mtm}{Main Theorem}
\newtheorem*{atm}{Another Presentation}
\newlist{lmlist}{enumerate}{1} \setlist[lmlist]{label={\rm(\roman{lmlisti})}, ref=\thelm.(\roman{lmlisti}),noitemsep} \Crefname{lmlisti}{Lemma}{Lemmas}


%Page layout
\textwidth 16cm 
\textheight 22cm 
\headheight 0.5cm 
\evensidemargin 0.3cm 
\oddsidemargin 0.2cm

\newcommand{\Max}{\mathop{\mathrm{Max}}\nolimits}
\newcommand{\proj}{\mathop{\mathrm{proj}}\nolimits}
\newcommand{\Card}{\mathop{\mathrm{Card}}\nolimits}
\newcommand{\Ker}{\mathop{\mathrm{Ker}}\nolimits}
\newcommand{\Cent}{\mathop{\mathrm{Cent}}\nolimits}
\newcommand{\E}{{\mathrm{E}}}
\newcommand{\GG}{{\mathrm{G}}}
\newcommand{\Um}{\mathop{\mathrm{Um}}\nolimits}
\newcommand{\St}{\mathop{\mathrm{St}}\nolimits}
\newcommand{\Sp}{\mathop{\mathrm{Sp}}\nolimits}
\newcommand{\Ep}{\mathop{\mathrm{Ep}}\nolimits}
\newcommand{\GL}{\mathop{\mathrm{GL}}\nolimits}
\newcommand{\SL}{\mathop{\mathrm{SL}}\nolimits}
\newcommand{\Kt}{\mathop{\mathrm{K_2}}\nolimits}
\newcommand{\Ko}{\mathop{\mathrm{K_1}}\nolimits}
\newcommand{\Ho}{\mathop{\mathrm{H_1}}\nolimits}
\newcommand{\Ht}{\mathop{\mathrm{H_2}}\nolimits}
\newcommand{\epi}{\twoheadrightarrow}
\newcommand{\sgn}{\mathrm{sgn}}
\newcommand{\eps}[1]{\varepsilon_{#1}}
\newcommand{\sign}[1]{\mathrm{sign}(#1)}
\newcommand{\lan}{\langle}
\newcommand{\ran}{\rangle}
\newcommand{\inv}{^{-1}}
\newcommand{\ur}[1]{\!\,^{(#1)}U_1}
\newcommand{\ps}[1]{\!\,^{(#1)}\!P_1}
\newcommand{\ls}[1]{\!\,^{(#1)}\!L_1}
\newcommand{\rA}{\mathsf{A}}
\newcommand{\rB}{\mathsf{B}}
\newcommand{\rC}{\mathsf{C}} 
\newcommand{\rD}{\mathsf{D}} 
\newcommand{\rE}{\mathsf{E}}
\newcommand{\rF}{\mathsf{F}}
\newcommand{\rG}{\mathsf{G}}
\DeclareRobustCommand{\VAN}[2]{#1}
\newcommand{\catname}[1]{{\normalfont\textbf{#1}}}

\renewcommand{\labelenumi}{\theenumi{\rm)}}
\renewcommand{\theenumi}{\alph{enumi}}

\newcommand\restr[2]{{\left.\kern-\nulldelimiterspace #1 \vphantom{\big|}\right|_{#2}}}
\newcommand*\MapsTo{\xrightarrow[\raisebox{0.25 em}{\smash{\ensuremath{\sim}}}]{}}  

\title{On centrality of orthogonal $\Kt$}
\keywords {Steinberg groups, $K_2$-functor. {\em Mathematical Subject Classification (2010):} 19C09}

\author {Andrei Lavrenov}
\email {avlavrenov {\it at} gmail.com}

\author{Sergey Sinchuk}
\email{sinchukss {\it at} gmail.com}

\date {\today}
\thanks{Authors of the present paper acknowledge the financial support from Russian Science Foundation grant 14-11-00297}

\begin{document}

\begin{abstract} We give a short uniform proof of centrality of $\mathrm K_2(\Phi, R)$ for all simply-laced root systems $\Phi$ of rank $\geq 3$.
\end{abstract}

\maketitle

%\begin{thebibliography}{99}
%\bibitem{Lav2} A.~Lavrenov, A local--global principle for symplectic $\mathrm K_2$, {\it to appear}.
%\bibitem{Rao} *Statement of Ravi Rao's hypothesis
%\bibitem{Swa1} *Swan's relative Steinberg group 1
%\bibitem{Swa2} *Swan's relative Steinberg group 2
%\bibitem{Swa3} *Swan's relative Steinberg group 3
%\bibitem{Tul0} *Tulenbaev's paper on almost commutative rings (cite somewhere)
%\end{thebibliography}

\section*{Introduction}
The aim of this paper is to establish Quillen's local-global principle and the centrality of even orthogonal $\Kt$ for arbitrary commutative ring $R$.

Recall that to each reduced irreducible root system $\Phi$ and a commutative ring $R$ one can associate a split simple group $\GG(\Phi, R)$ called a \emph{Chevalley group}.
The {\it elementary group} $\E(\Phi, R)$ is defined as the subgroup of $\GG(\Phi, R)$ generated by elementary root unipotents $t_\alpha(\xi)$, $\xi\in R$, $\alpha\in \Phi$, see~\cite{Ta, St78}.
By definition, the Steinberg group $\St(\Phi, R)$ is presented by generators $x_\alpha(\xi)$, which model the unipotents $t_\alpha(\xi)$, and Steinberg relations (see~\cref{sec:prelim}). 
Thus, there is a well-defined map $\phi\colon\St(\Phi, R)\to \GG(\Phi, R)$ sending $x_\alpha(\xi)$ to $t_\alpha(\xi)$ whose image is $\E(\Phi, R)$.

By a theorem of Taddei (see~\cite{Ta}) the elementary group $\E(\Phi, R)$ is a normal subgroup of $\GG(\Phi, R)$ if $\Phi$ has rank at least $2$.
Thus, similarly to the definition of algebraic $\mathrm{SK}_1$ and $\Kt$-functors, one can define the groups $\Ko(\Phi, R)$, $\Kt(\Phi, R)$ as the cokernel and kernel of $\phi$:
$$\xymatrix{1 \ar[r] & \Kt(\Phi, R) \ar[r] & \St(\Phi, R) \ar[r] & \GG(\Phi, R) \ar[r] & \Ko(\Phi, R) \ar[r] & 1.}$$

It is a classical theorem of M.~Kervaire and R.~Steinberg that the stable Steinberg group $\St(R)$ is the universal central extension of $\E(R)$, see \cite[Theorem~III.5.5]{Kbook}.
The unstable analogue of the ``universality'' part of this theorem has been known since 1970's for $\Phi$ with of $\ell \geq 5$ (or even $\ell \geq 4$ for $\Phi=\rA_\ell, \rB_\ell, \rC_\ell$), see \cite{St1, KaS}.
However, until recently, the analogue of the ``centrality'' part was only known in the special case $\Phi=\rA_\ell$, $\ell\geq 3$ see~\cite[Corollary~2]{vdK}.
The best known result for other classical $\Phi$ was that the centrality holds ``in the stable range'',
i.\,e. when the dimension of $R$ is suffiently small as compared to the rank of $\Phi$, see~\cite[Corollary~3.4]{St78}.

Recently the authors of this paper have established the centrality of $\Kt(\rC_\ell, R)$ ($\ell\geq 3$) and $\Kt(\rE_\ell, R)$ ($\ell=6,7,8$) for arbitrary commutative ring $R$, see \cite{Lav, SCh}.
In turn, the key result of this paper is the following theorem.
\begin{tm}[``Centrality of $\Kt$''] \label{tm:centrality}  Let $R$ be an arbitrary commutative ring.
Then for $\ell\geq 4$ the group $\Kt(\rD_\ell, R)$ is a central subgroup of $\St(\rD_\ell, R)$. \end{tm}

Centrality of $\Kt$ is a corollary of the following analogue of Quillen's local-global principle,
very much in the same fashion as the normality theorem for the elementary subgroup is a corollary of the local-global principle for $\Ko$, cf.~\cite[Theorem~3.1]{Basu05}.
\begin{tm}[``Quillen's LG-principle for $\Kt$''] \label{tm:lg-principle} 
Let $\Phi$ be an irreducible simply-laced root system of rank $\geq 3$ and $R$ be arbitrary commutative ring.
Then an element $g\in\St(\Phi,\,R[X])$ satisfying $g(0)=1\in\St(\Phi,\,R)$ is trivial in $\St(\Phi,\,R[X])$
if and only if the images $\lambda_M^*(g)\in\St(\Phi,\,R_M[X])$ of $g$ under localisation maps $\lambda_M\colon R\to R_M$ are trivial for all maximal ideals $M\trianglelefteq R$. \end{tm}

In the special case $\Phi=\rA_\ell$, $\ell\geq 4$ the assertion of \cref{tm:lg-principle} was obtained by Tulenbaev, see~\cite[Theorem~2.1]{Tul}.
Despite the fact that the key ingredient in Tulenbaev's proof (``another presentation'' of the Steinberg group) is available for $\Phi=\rA_3$, Tulenbaev's proof does not work in this case for technical reasons.
On the other hand, in~\cite{SCh} the second-named author developed a patching technique which allowed him to deduce \cref{tm:lg-principle} from Tulenbaev's result in another special case $\Phi=\rE_\ell$, $\ell=6,7,8$.
Thus, \cref{tm:lg-principle} contains two cases that have not been previously known: $\Phi=\rA_3$ and $\Phi=\rD_\ell$, $\ell\geq 4$.
We also note that recently the first-named author has obtained an analogue of \cref{tm:lg-principle} for $\Kt(\rC_\ell, R)$ under the assumption $\ell\geq 3$.

The rest of the paper is organised as follows. 
After introducing the preliminaries we explain in \cref{sec:patching} how the patching technique of~\cite{SCh} allows one to reduce \cref{tm:lg-principle} to the special case $\Phi=\rA_3$.
Next, in \cref{sec:yap} we describe a new ``transpose-symmetric'' presentation for the relative linear Steinberg group.
Finally, this presentation is used in~\cref{sec:lgp} to prove \cref{tm:lg-principle} in the case $\Phi=\rA_3$.
%It mimics the standart vector representation of $\SL(n,\,R)$, therefore in the first section we use representations and corresponding vector notation. 
%Finally, in \cref{sec:patching} we derive our principal results from this special case and the amalgamation theorem for relative Steinberg groups from~\cite{SCh}.
%There we switch to the root notation.

\section{Preliminaries} \label{sec:prelim}
Throughout this paper $R$ denotes an associative commutative ring with identity.
All commutators are left-normed, i.\,e. $[x,\,y]=xyx\inv y\inv$. 
We denote by $R^n$ the free $R$-module with basis $e_1,\ldots,e_n$ and by $\Um(n,\,R)$ the subset of unimodular columns $v\in R^n$ 
whose entries generate $R$ as an ideal.

%Tulenbaev used {\it another presentation} for a relative Steinberg group to prove a local--global
%principle for a linear Steinberg group~\cite{Tul}. 
%This relative another presentation was inspired by the {\it absolute} one of van der Kallen~\cite{vdK}. 
%However, Tulenbaev's presentation can only be used for $\St(l+1,\,R)=\St(\mathrm A_l,\,R)$, $l\geq4$.

%Tulenbaev's result can be applied to the subsystems of type $\mathrm A_4$ inside $\mathrm E_l$ and ``glued'' 
% to get a local--global principle for Steinberg groups of type $\mathrm E_l$~\cite{SCh}. 
%To use this glueing method for groups of type $\mathrm D_l$ as well one has to work with subsystems 
%of type $\mathrm A_3$ instead, and thus extend Tulenbaev's result to the case of $\mathrm A_3$.

%Recently, the local--global principle was obtained for Steinberg groups of type $\mathrm C_l$ starting from rank 3~\cite{Lav2}. 
%That is another motivation to consider the linear Steinberg group in rank 3.

First of all, recall that for a simply laced root system $\Phi$ of rank $\geq 2$ the Steinberg group $\St(\Phi, R)$ 
is defined by generators $x_\alpha(r)$, $\alpha\in\Phi$, $r\in R$ and the following relations:
\begin{align}
 x_\alpha(r) x_\alpha(s) & = x_\alpha(r+s) & \nonumber \\
 [x_\alpha(r),  x_\beta(s)] & = x_{\alpha+\beta}(N_{\alpha,\beta} rs) &\ \alpha,\beta\in \Phi,\ \alpha+\beta\in\Phi \nonumber \\
 [x_\alpha(r),  x_\beta(s)] & = 1 &\ \alpha,\beta\in \Phi,\ \alpha+\beta\not\in\Phi \nonumber
\end{align}
Here $N_{\alpha,\beta}$ are the structure constants of $\Phi$ which are equal to $\pm 1$.
In the case $\Phi=\rA_\ell$ we use a more common notation for root unipotents: $x_{ij}(r)$, $1\leq i\neq j\leq n$, $r\in R$,
\setcounter{equation}{0}
\renewcommand{\theequation}{S\arabic{equation}}
\begin{align}
x_{ij}(r)x_{ij}(s)      & = x_{ij}(r+s),& \label{add0}\\
[x_{ij}(r),\,x_{hk}(s)] & = 1,& \text{ for }h\neq j,\ k\neq i, \label{ccf1}\\
[x_{ij}(r),\,x_{jk}(s)] & = x_{ik}(rs)& \label{ccf2}.
\end{align}
We denote by $\E(n, R)=\E(\rA_{n-1}, R)$ the elementary subgroup of $\SL(n, R)$, i.\,e. the subgroup generated by elementary transvections $t_{ij}(r)=1+r \cdot e_{ij}$, $1\leq i\neq j\leq n$, $r\in R$.
Here $1$ denotes the identity matrix and $e_{ij}$ is the standard matrix unit. The natural projection $\phi\colon\St(n,\,R)\rightarrow\E(n,\,R)$ sends $x_{ij}(r)$ to $t_{ij}(r)$.

In \cite{Sus} Suslin showed for $n\geq 3$ that the elementary group $\E(n, R)$ coincides with the subgroup of $\GL(n,\,R)$ generated by matrices
of the form $t(u,\,v)=1+uv^t$ where $u\in\Um(n,\,R)$, $v\in R^n$ and $u^tv=0$. Here $u^t$ stands for the transpose of $u$.
This result clearly implies that $\E(n,\,R)$ is normal inside $\GL(n,\,R)$.
Afterwards, W.~van~der~Kallen developed Suslin's ideas and showed for $n\geq4$ that the Steinberg group $\St(n,\,R)$ is isomorphic to the group presented by generators
$$\{X(u,\,v)\mid u\in\Um(n,\,R),\ v\in R^n,\ u^tv=0\}$$ and the following list of relations (see~\cite[Theorem~1]{vdK}):
\setcounter{equation}{0} \renewcommand{\theequation}{K\arabic{equation}}
\begin{align}
&X(u,\,v)X(u,\,w)=X(u,\,v+w), \label{add1} \\
&X(u,\,v)X(u',\,v')X(u,\,v)\inv=X(t(u,\,v)u',\,t(v,\,u)\inv v'). \label{conj1}
\end{align}
This presentation clearly implies that $\phi\colon\St(n,\,R)\rightarrow\E(n,\,R)$ is a central extension.
%Our notation for the generators of $\St(n, R)$ differs from van der Kallen's, e.\,g.
Notice that we parametrize elements $X(u, v)$ with two columns rather than with one column and one row as it is done in~\cite{vdK}.
Our generator $X(u,v)$ corresponds to van der Kallen's generator $X(u, v^t)$.

Recall that an ideal $I\trianglelefteq R$ is called a \emph{splitting ideal} if the canonical projection $R \twoheadrightarrow R/I$ splits as a unital ring morphism, i.\,e. one has $\pi\sigma=1$.
$$\xymatrix{0\ar@<0.1ex>[r]^{}&I\ar@<0.1ex>[r]^{}&R\ar@<0.5ex>[r]^{\pi}&R/I\ar@<0.5ex>[l]^{\sigma}\ar@<0.1ex>[r]^{}&0}$$
For a splitting ideal $I\trianglelefteq R$ the \emph{relative Steinberg group} $\St(\Phi, R, I)$ can be defined as the kernel $\Ker\big(\St(\Phi,\,R)\epi\St(\Phi,\,R/I)\big)$.

For arbitrary $I$ the relative Steinberg group $\St(\Phi, R, I)$ is no more a subgroup of $\St(\Phi, R)$ and should be defined as a central extension of the above kernel.
For the purposes of the present text it suffices to consider only relative Steinberg groups corresponding to splitting ideals.
For more information regarding the general case we refer the reader to \cite[Section~3]{SCh}.

\section{Reduction to the case \texorpdfstring{$\rA_3$}{A3}} \label{sec:patching}
First, we make the following definition inspired by Lemmas~2.3,~3.2 of~\cite{Tul}.

\begin{df} \label{df:tlp}
We say that the Steinberg group functor $\St(\Phi, -)$ satisfies {\it Tulenbaev lifting property} if for any ring $R$ and any non-nilpotent element $a \in R$
there exists a map $T_\Phi$ which fits into the following commutative diagram.
\numberwithin{equation}{section}
\setcounter{equation}{1}
\begin{equation} \xymatrix{\St(\Phi,\,R\ltimes XR_a[X],\,XR_a[X]) \ar@{^{(}->}[r] \ar[d]_-{\lambda_a^*} & \St(\Phi, R \ltimes XR_a[X]) \ar[d]_-{\lambda_a^*} \\
                           \St(\Phi,\,R_a[X],\,XR_a[X]) \ar@{-->}[ru]_-{T_\Phi} \ar@{^{(}->}[r] & \St(\Phi, R_a[X])} \label{eq:diagram} \end{equation}
Here $\lambda_a^*$ denotes the morphism of Steinberg groups induced by the morphism $\lambda_a\colon R \to R_a$ of localisation by the powers of $a$.                           
Notice that $XR_a[X]$ is a splitting ideal for both $R_a[X]$ and $R \ltimes XR_a[X]$.
\end{df}
It is clear that the image of $T_\Phi$ is contained in the image of the top arrow, therefore, Tulenbaev lifting property implies that the relative Steinberg groups in the left-hand side of the diagram are isomoprhic.

We claim that the main results of this paper follow from Tulenbaev's lifting property.
\begin{prop} Consider the following statements:
\begin{enumerate}
 \item the Steinberg group functor $\St(\Phi, -)$ satisfies lifting property~\eqref{eq:diagram};
 \item a dilation principle holds for $\St(\Phi, -)$, i.\,e. if $g\in\St(\Phi, A[X], XA[X])$ is such that $\lambda_a^*(h) = 1 \in \St(\Phi, R_a[X])$ then
       for sufficiently large $n$ one has $ev_{X = a^n Y}^*(h) = 1 \in \St(\Phi, A[Y])$, where $ev_{X=a^n Y}$ denotes the unique $R$-algebra morphism $R[X]\to R[Y]$ sending $X$ to $a^nY$;
 \item Quillen's local global principle for $\Kt$ holds for $\St(\Phi, -)$;
 \item the centrality of $\Kt$ holds for $\St(\Phi, -)$.
\end{enumerate}
Then for any irreducible $\Phi$ of rank $\geq 2$ there are implications $a) \implies b) \implies c) \implies d)$. \end{prop}
\begin{proof} The specified implications are contained in the proofs of Lemma~15, Theorem~2 and Theorem~1 of \cite{SCh}, respectively. \end{proof}
     
In order to construct the arrow $T_\Phi$ fitting in the diagram~\eqref{eq:diagram} one needs to have some direct description of the relative Steinberg group $\St(\Phi, R, I)$.
For example, in~\cite{Tul} Tulenbaev uses a variant of of W.~van~der~Kallen's ``another presentation'' for this purpose. The details of this approach are outlined in the next section.
On the other hand, in~\cite{SCh} the second-named author obtained a theorem which asserts that the relative Steinberg group $\St(\Phi, R, I)$ can be presented as an amalgamated product of
several copies of relative Steinberg groups of smaller rank. 

To formulate this theorem we will need several technical definitions.
For a simply laced root system $\Phi$ we denote by $\catname{Subsys}(\Phi)$ the category of root subsystems of $\Phi$.
Its objects are all root subsystems $\Psi\subseteq \Phi$ and its morphisms are in one-to-one correspondence with inclusions of the form $\Psi_1\subseteq \Psi_2$.
It is clear that the relative Steinberg group can be turned into a functor $\St(-, R, I)\colon \catname{Subsys}(\Phi) \rightarrow\catname{Groups}.$

We denote by $\mathcal{G}_{1,n}$ the full subcategory of $\catname{Subsys}(\Phi)$ whose objects are root subsystems $\Psi\subseteq \Phi$ having type $\rA_1$ or $\rA_n$. 
It is easy to see that $\mathcal{G}_{1,n}$ is a directed bipartite graph whose only nonidentical morphisms are inclusions of the form $i_{\alpha, \Psi}\colon \{\alpha, -\alpha\} \subseteq \Psi \cong \rA_3$, $\Psi \subseteq \Phi$.
\begin{tm*} \label{tm:relPres} For a simply laced irreducible root system $\Phi$ of rank $\geq 3$
the group $\St(\Phi, R, I)$ is isomorphic as an abstract group to $\varinjlim_{\mathcal{G}_{1,3}}\St(-, R, I)$. \end{tm*}
\begin{proof} See~\cite[Theorem~9]{SCh} \end{proof}

The above theorem allows one to construct the arrow $T_\Phi$ by means of the universal property of colimits.
More precisely, one needs to construct the maps $T_{\Psi}$ for subsystems $\Psi\subseteq\Phi$ of type $\rA_3$ and then check that
relations $T_\Psi \circ i_{\alpha, \Psi} = T_{\Psi'} \circ i_{\alpha, \Psi'}$ hold for all $\Psi$, $\Psi'$ of type $\rA_3$ containing a common root $\alpha$
(here $i_{\alpha, \Psi}$ denotes the obvious morphism $\St(\{\alpha, -\alpha\}, R, I) \to \St(\Psi, R, I)$).
The latter check is relatively formal and, essentially, has already been made in~\cite{SCh} 
(see the proof of Lemma~14 and notice that everywhere in it $\rA_4$'s can be replaced with $\rA_3$'s).
On the other hand, constructing $T_\Psi$ for $\Psi=\rA_3$ is not at all obvious and will be done in the section to follow.

Summing up the above discussion, we have shown that both \cref{tm:centrality} and \cref{tm:lg-principle} follow from Tulenbaev's lifting property in the special case $\Phi=\rA_3$.

%This $T$ is an inverse to a map induced by localisation, so that constructing $\mathrm T$ may be treated
%as a proof of excision property for Steinberg group in this very particular situation.

%To prove the local--global principle for Steinberg groups of type $\mathrm D_l$ and centrality 
%of $\mathrm K_2(\mathrm D_l,\,R)$ we need such a map for $n=4$ (i.e., for the group of type $\mathrm A_3$).


%Since there are known counterexamples to centrality of $\Kt$ and, thus, to \cref{tm:lg-principle} for $\Phi$ of rank $2$ (see~\cite{W}), 
%this gives a complete answer to the question for which $\ell$ the local-global principle holds for $\Kt(\rC_\ell, R)$.

%The main idea of our proof of~\cref{tm:lg-principle} is to first settle the case $\Phi=\rA_3$ and then derive the other cases from it using the patching technique of~\cite{SCh}.

%Compare \cref{tm:lg-principle} with the local-global principle for $\Ko$ which states that an element of the congruence subgroup $g\in \GG(\Phi,\,R[X], XR[X])$ is elementary
%(i.\,e. lies in $\E(\Phi, R[X])$) if and only if $\lambda_M^*(g)=1\in\E(\Phi,\,R_M[X])$ for all maximal ideals $M \trianglelefteq R$.


\section{Yet another presentation for the relative Steinberg group.} \label{sec:yap}

In order to construct the arrow $T$ Tulenbaev uses the following presentation for the relative Steinberg group, see~\cite[Proposition~1.6]{Tul}. 
%Assume that $I\trianglelefteq R$ such that a short exact sequence
%\renewcommand{\theequation}{$\dagger$}
\setcounter{lm}{1}
\begin{prop}\label{prop:TulPres}
For a splitting ideal $I$ and $n\geq 4$ the group $\St(n,\,R,\,I)$ is isomorphic to the group defined by generators
$$\{X(u,\,v)\mid u\in\E(n,\,R)e_1,\ v\in I^n,\ u^tv=0\}$$ and the following set of relations:
\setcounter{equation}{0}
\renewcommand{\theequation}{T\arabic{equation}}
\begin{align}
&X(u,\,v)X(u,\,w)=X(u,\,v+w), \label{add2}\\
&X(u,\,v)X(u',\,v')X(u,\,v)\inv=X(t(u,\,v)u',\,t(v,\,u)\inv v'), \label{conj2}  \\
&X(ur+w,\,v)=X(u,\,vr)X(w,\,v)\,\text{ for }r\in R,\ (u,\,w)\in\Um_{n\times2}(R) \label{add3}.
\end{align}
Moreover, the group $\St(n, R)$ decomposes into the semidirect product $\St(n,\,R, I)\rtimes \St(n,\,R/I)$.
\end{prop}
Here $\Um_{n\times2}(R)$ denotes the set of $n\times2$ \emph{unimodular matrices}, i.\,e. matrices $M$ such that there exists a $2\times n$ matrix $N$ satisfying $NM=\begin{pmatrix}1&0\\0&1\end{pmatrix}$.
In the definition of the semidirect product the group $\St(n, R/I)$ acts on $\St(n, R, I)$ by conjugation via the splitting map, i.\,e. ${}^g X(u,\,v) = X(\phi(\sigma(g))u, \phi(\sigma(g))^*v)$ for $g \in \St(n, R/I)$.

\begin{proof} For the sake of completeness we present below a more detailed version of Tulenbaev's proof.
Tulenbaev constructs a pair of mutually inverse maps
$$\xymatrix{ \St(n, R, I) \rtimes \St(n, R/I) \ar@<0.6ex>[r]^-\varphi &  \ar[l]^-\psi \St(n, R).}$$

Define the map $\psi$ by $\psi(x_{ij}(\xi)) = (X(e_i, (\xi - \pi(\xi)e_j),\ x_{ij}(\pi(\xi)))$. 
We have to check that $\psi$ preserves relations \eqref{add0}--\eqref{ccf2}.
For example, let us check \eqref{ccf2}. 
First of all, recall that in any semidirect product of groups one can compute the commutator of two elements in the following way:
$$[(a, b), (c, d)] = (a \cdot {}^bc \cdot {}^{bdb^{-1}}a^{-1} \cdot {}^{[b, d]}c^{-1} , [b, d]).$$
For $\xi\in R$ set $\xi' = \xi - \pi(\xi)$.
Specializing $a = X(e_i, \xi' e_j)$, $b = x_{ij}(\pi(\xi))$, $c = X(e_j, \eta'e_k)$, $d = x_{jk}(\pi(\eta))$ and using the definition of the conjugation action 
one computes the commutator $[\psi(x_{ij}(\xi), \psi(x_{jk}(\eta)]$ as follows:
\setcounter{equation}{2}
\begin{equation} \left(X(e_i, \xi' e_j) \cdot X(e_j + \pi(\xi)e_i, \eta'e_k) \cdot X(e_i, \pi(\eta)\xi' e_k - \xi' e_j) \cdot X(e_j, -\eta'e_k),\ x_{ik}(\pi(\xi\eta))\right). \label{eq:proofS3}\end{equation}
\setcounter{lm}{3}
After applying additivity relations \eqref{add2}, \eqref{add3} and moving the factor $X(e_i, -\xi' e_j)$ to the left hand side of the formula by means of relation \eqref{conj2}
one gets the following expression:
$$(X(e_j, \eta'e_k) \cdot X(e_i, (\xi' \eta' + \pi(\xi)\eta' + \pi(\eta)\xi')e_k \cdot X(e_j, -\eta'e_k),\ x_{ik}(\pi(\xi\eta)).$$
Since $\xi'\eta' + \pi(\xi)\eta' + \pi(\eta)\xi' = \xi\eta - \pi(\xi\eta)$ the above expression clearly simplifies to $\psi(x_{ik}(\xi\eta))$, as claimed.
Verification of the fact that $\psi$ preserves relations \eqref{add0}, \eqref{ccf1} is similar to the above computation but is easier.
In particular, one does not need to use relation~\eqref{add3}.

It remains to check that the map $\varphi$ given by $\varphi\left((X(u, v), 1)\right) = X(u, v),$ $\varphi\left((1, x_{ij}(\xi))\right) = X(e_i, \sigma(\xi)e_j)$
is well-defined and inverse to $\psi$.
\end{proof}

\begin{rk} \label{rk:T123'} Notice that from the proof of the above proposition it follows that one could replace the family of relations \eqref{add3}
in the definition of $\St(n, R, I)$ with the following smaller family:
\setcounter{equation}{2} \renewcommand{\theequation}{T\arabic{equation}'}
\begin{equation} X(e_ir+e_j,\,e_k)=X(e_i,\,rae_k)X(e_j,\,ae_k)\,\text{ for }r\in R,\ a\in I\text{ and } i,j,k\text{ distinct}. \label{add3'} \end{equation} 
This follows from the fact that only this smaller family of relations was actually used in the construction of the map $\psi$
(more precisely, it was used to simplify the second factor of~\eqref{eq:proofS3}). \end{rk}

Both ``another presentations'' of van der Kallen and Tulenbaev are given in terms of generators parametrised by pairs of vectors,
 where the first one is ``good'' in some sense  (i.\,e. unimodular or a column of an elementary matrix) while the second one is arbitrary.
It is easy to formulate the additivity property in the second argument for these generators (cf. \eqref{add1}, \eqref{add2}),
while it is not so easy when it comes to the additivity in the first argument (cf. \eqref{add3}).
%Such presentations allowed them to formulate easily an additivity by the second argument, but one has to write something similar to T3 instead of additivity by the first one.

The construction of the homomorphism $T\colon\St(n,\,R_a[X],\,XR_a[X])\rightarrow\St(n,\,R\ltimes XR_a[X])$
amounts to choosing certain elements in the group $\St(n,\,R\ltimes XR_a[X])$ and proving that these elements satisfy relations \eqref{add2}--\eqref{add3}, see~\cite[Lemmas~1.2 and~1.3\,c)]{Tul}.
The main problem with this recipe is that the assumption $n\geq5$ is essential to check that the relation~\eqref{add3} holds.

It is not possible to choose arbitrary orthogonal vectors $u$, $v$ as parameters for the generators because in this situation it is not even known that $t(u,\,v)\in\E(n,\,R)$, see~\cite{Rao}. 

To generalise Tulenbaev's results for $n=4$ we use a more symmetric presentation with two types of generators: 
$F(u,\,v)$ with $u$ good and $v$ arbitrary and $S(u,\,v)$ with $u$ arbitrary and $v$ good.
The generators $F(u,\,v)$ are additive in the second component, while $S(u,\,v)$ are additive in the first one.
When $u$ and $v$ are both good we require that these two generators coincide.
More formally, we make the following definition.

\begin{df}
For $I\trianglelefteq R$ and $n\geq4$ define $\St^*(n,\,R,\,I)$ to be the group with the set of generators
\begin{multline*}
\{F(u,\,v)\mid u\in\E(n,\,R)e_1,\ v\in I^n,\ u^tv=0\}\,\cup\{S(u,\,v)\mid u\in I^n,\ v\in\E(n,\,R)e_1,\ u^tv=0\}
\end{multline*}
subject to relations
\setcounter{equation}{0}
\renewcommand{\theequation}{R\arabic{equation}}
\begin{align}
&F(u,\,v)F(u,\,w)=F(u,\,v+w), \label{add4}\\
&S(u,\,v)S(w,\,v)=S(u+w,\,v), \label{add5}\\
&F(u,\,v)F(u',\,v')F(u,\,v)\inv=F(t(u,\,v)u',\,t(v,\,u)\inv v'), \label{conj3} \\
&F(u,\,va)=S(ua,\,v),\ \text{for all}\ a\in I,\,(u,\,v^t)=(M e_1,\,e_2^t M\inv),\,M\!\in\E(n,\,R) \label{coinc}.
\end{align}
\end{df}
Notice that we only described how a generator $F(u, v)$ acts via conjugation on another generator $F(u',v')$ and have omitted three similar relations involving generators $S(u,v)$ of the second type.
The reason for this is the following lemma which asserts that the ``missing'' relations will follow automatically from \eqref{add4}--\eqref{coinc}.
\begin{lm}
\label{allyouneedisf}
Denote by $\phi\colon\St^*(n,\,R,\,I)\rightarrow\E(n,\,R)$ the natural map sending $F(u,\,v)\mapsto t(u,\,v)$ and $S(u,\,v)\mapsto t(u,\,v)$.
Then the following facts are true.
\begin{enumerate}
\item $\St^*(n,\,R,\,I)$ is generated as an abstract group by the set of elements 
      $$\{F(u,\,va)\mid a\in I,\ (u,\,v^t)=(Me_1,\,e_2^tM\inv),\,M\in\E(n,\,R)\};$$
\item for any $g\in\St^*(n,\,R,\,I)$ one has 
      \setcounter{equation}{2} \renewcommand{\theequation}{R\arabic{equation}'}
      \begin{equation} gF(u,\,v)g\inv=F(\phi(g)u,\,\phi(g\inv)^tv); \end{equation}
\item for any $g\in\St^*(n,\,R,\,I)$ one has
       \setcounter{equation}{2} \renewcommand{\theequation}{R\arabic{equation}''}
      \begin{equation} gS(u,\,v)g\inv=S(\phi(g)u,\,\phi(g\inv)^tv); \end{equation}
\item there is a ``transpose automorphism'' defined on $\St^*(n, R, I)$ satisfying
      $$F(u,\,v)^t=S(v,\,u),\quad S(u,\,v)^t=F(v,\,u).$$
\end{enumerate} \end{lm}

\begin{proof}
Let $F(u,\,v)$ be an arbitrary generator of the first type and let $M\in\E(n,\,R)$ be such that $F(u,\,v)=F(Me_1,\,M^*\tilde v)$. 
Here we denote by $M^*$ the contragradient matrix, $M^*=(M\inv)^t$. 
By \eqref{add4} we have $$F(u,\,v)=\prod\limits_{k\neq1}F(Me_1,\,M^*e_k\tilde v_k)$$
where $\tilde v_k$ stands for $k$-th coordinate of $\tilde v=\sum e_i\tilde v_i$.
Applying relations \eqref{add5}, \eqref{coinc} we get
$$S(u,\,v)=\prod\limits_{k\neq1}S(Ne_k\tilde u_k,\,N^*e_1)=\prod\limits_{k\neq1}F(Ne_k,\,N^*e_1\tilde u_k)$$
and, thus, obtain $(\mathrm{a})$. Obviously, $(\mathrm{b})$ follows from $(\mathrm{a})$. 
To prove $(\mathrm{c})$ it suffices to show that
$$F(u,\,v)S(u',\,v')F(u,\,v)\inv=S(t(u,\,v)u',\,t(u,\,v)^*v').$$
For $S(u',\,v')=\prod_{k\neq 1} F(Ne_k,\,N^*e_1\tilde u_k)$ we have
\begin{multline*}
F(u,\,v)S(u',\,v')F(u,\,v)\inv=F(u,\,v)\prod F(Ne_k,\,N^*e_1\tilde u_k)F(u,\,v)\inv=\\
=\prod F(t(u,\,v)Ne_k,\,t(u,\,v)^*N^*e_1\tilde u_k)=\\
=\prod S(t(u,\,v)Ne_k\tilde u_k,\,t(u,\,v)^*N^*e_1)=S(t(u,\,v)u',\,t(u,\,v)^*v').
\end{multline*}
Finally, $(\mathrm{d})$ follows from $(\mathrm{c})$.
\end{proof}

Our next goal is to show for a splitting ideal $I\trianglelefteq R$ that the group $\St^*(n,\,R,\,I)$ is isomorphic to $\St(n,\,R,\,I)=\Ker\big(\St(n,\,R)\epi\St(n,\,R/I)\big)$.
With this end we construct two mutually inverse homomorphisms
$$\xymatrix{\St^*(n,\,R,\,I)\ar@<0.5ex>[r]^{\iota}&\St(n,\,R,\,I)\ar@<0.5ex>[l]^{\kappa}.}$$

In order to construct the arrow $\kappa$ we use the presentation from \cref{prop:TulPres}.
We set $$\kappa(X(u,v)) = F(u,v),\ u\in \E(n, R)e_1,\ v\in I^n.$$ 
In view of \cref{rk:T123'} it suffices to check that $F(u,v)$ satisfy the relations \eqref{add2}, \eqref{conj2}, \eqref{add3'}.
The validity of relations \eqref{add2}, \eqref{conj2} is obvious while \eqref{add3'} follows from \eqref{add5} and \eqref{coinc}, indeed:
$$ F(e_i, rae_k) F(e_j, ae_k) = S(rae_i, e_k) S(ae_j, e_k) = S(a(re_i+e_j), e_k) = F(e_ir+e_j, ae_k). \qedhere $$

As for the map $\iota$, we firsty define it as a homomorphism to the absolute group.
Then we prove that the image of $\iota$ is contained in $\St(n, R, I)$.
%$$\xymatrix{\St^*(n,\,R,\,I)\ar@<0.0ex>[r]^{\iota}&\St(n,\,R).}$$
Clearly, $\iota$ should map the elements $F(u,\,v)$ to van der Kallen's generators $X(u,\,v)$.
We should find the images for the elements $S(u,\,v)$ as well.
First of all, recall that van der Kallen in~\cite[3.8--3.10]{vdK} constructs the following elements:
$$x(u,\,v)\in\St(n,\,R),\ u^tv=0,\ u_i=0\,\text{ or }\,v_i=0\,\text{ for some }\,1\leq i\leq n.$$
The following definition is the ``transpose'' of \cite[3.13]{vdK}.
\begin{df} For $u\in R^n$, $v\in\E(n,\,R)e_1$, $u^tv=0$, consider the set $\overline Y(u,\,v)\subseteq\St(n,\,R)$
 consisiting of all elements $y\in\St(n,\,R)$ that admit a decomposition $y=\prod x(w^k,\,v)$, 
 where $\sum u^k=u$ and $u^k= (e_pv_q-e_qv_p)c^k$ for some $c^k\in R$, $1\leq p\neq q\leq n$. \end{df}

Since columns of elementary matrices are unimodular, one can find $w\in R^t$ such that $w^tv=1$.
It is not hard to check that \setcounter{equation}{6} \setcounter{lm}{7}
\begin{equation} (w^t\cdot v)\cdot u = \sum_{p<q}u_{pq},\text{ where }u_{pq} = (e_pv_q - e_qv_p)\cdot  (u_pw_q - u_qw_p)\in{}\!R^n, \label{eq:canonical}\end{equation}
therefore $\overline Y(u,\,v)$ is not empty (cf.~\cite[3.1--3.2]{vdK}).
Obviously, for $x\in\overline Y(u,\,v)$ and $y\in\overline Y(w,\,v)$ one has $xy\in\overline Y(u+w,\,v)$.

Repeating~\cite[3.14--3.15]{vdK} verbatim one shows the following.
\begin{lm} For $g\in\St(n,\,R)$ one has
\begin{enumerate} \item $g\overline Y(u,\,v)g\inv\subseteq\overline Y(\phi(g)u,\,\phi(g)^*v)$;
                  \item $\overline Y(u,\,v)$ consists of exactly one element. \end{enumerate} \end{lm}

We denote the only element of $\overline Y(u,\,v)$ by $Y(u,\,v)$. 
Now we can finish the definition of $\iota$ by requiring that $\iota(S(u,\,v)) = Y(u,\,v)$ for $S(u,\,v)\in\St^*(n,\,R,\,I)$. 
To show that the map $\iota$ is well-defined we should check that elements $X(u,\,v)$ and $Y(u,\,v)$ satisfy relations \eqref{add4}--\eqref{coinc} 
(with $F$'s and $S$'s replaced with $X$'s and $Y$'s).
Only the relation \eqref{coinc} is not immediately obvious.
\begin{lm} \label{lm:XY} For $(u,\,v^t)=(Me_1,\,e_2^tM\inv)$, $M\in\E(n,\,R)$, $a\in I$ one has $X(u,\,va)=Y(ua,\,v)$. \end{lm}
\begin{proof}
In view of the above lemma we only need to show that $X(e_1,\,e_2a)=Y(e_1a,\,e_2)$.
This equality can be obtained by computing the commutator $[Y(-e_3,\,e_2),\,X(e_1,\,e_3a)]$ in two ways:
$$ Y(-e_3,\,e_2)X(e_1,\,e_3a)Y(-e_3,\,e_2)\inv\cdot X(e_1,\,-e_3a)=X(e_1,\,e_2a), \text{ and} $$
$$ Y(-e_3,\,e_2)\cdot\,X(e_1,\,e_3a)Y(e_3,\,e_2)X(e_1,\,e_3a)\inv=Y(e_1a,\,e_2). \qedhere $$
\end{proof}

Since $\pi^*(\iota(F(u, v))) = X(\pi(u),\,0)=1$ and $\pi^*(\iota(S(u,v)) = Y(0,\,\pi(v)) = 1$ we get that $\mathrm{Im}(\iota)\subseteq\Ker(\pi^*)=\St(n,\,R,\,I)$. 
It is clear that $\iota\kappa=\mathrm{id}$ hence $\kappa$ is injective.
On the other hand, $\kappa$ is surjective by Lemma~\ref{allyouneedisf}\,$(\mathrm a)$.
Thus, we have demonstrated the following result.

\begin{prop} \label{lm:map-iota}
 For a splitting ideal $I \trianglelefteq R$ and $n\geq 4$ the groups $\St^*(n,\,R,\,I)$ and $\St(n,\,R, I)$ are isomorphic.
% there exists a well-defined homomorphism $$\iota\colon\St^*(n,\,R,\,I)\rightarrow\St(n,\,R, I) \subseteq \St(n, R)$$ 
% satisfying the equalities $\iota (F(u,\,v)) = X(u,\,v)$ and $\iota (S(u,\,v)) = Y(u,\,v)$. 
\end{prop}
%\begin{proof}
%We already know that $\iota$ is well-defined. It suffices to check that the image of $\iota$ is contained in $\St(n,\,R,I)$.
%Denote the canonical projection $R\epi R/I$ by $\rho$.
%Clearly, the induced group morphism $\rho^*\colon\St(n,\,R)\rightarrow\St(n,\,R/I)$ sends $X(u,\,v)$ to $X(\rho(u),\,\rho(v))$ and $Y(u,\,v)$ to $Y(\rho(u),\,\rho(v))$. 
%Observe that $\E(n,\,R)\epi\E(n,\,R/I)$ is surjective, therefore for $x\in\E(n,\,R)e_1$ one has $\rho(x)\in\E(n,\,R/I)e_1$. 
%, as required. \end{proof}
 
\begin{comment}
We constructed $\iota$ for an arbitrary $I\trianglelefteq R$. For $I$ a splitting ideal we construct an inverse homomorphism
$$
\kappa\colon\St(n,\,R,\,I)\rightarrow\St^*(n,\,R,\,I).
$$
With this end we recall a presentation of $\St(n,\,R,\,I)$ as a group with action of $\St(n,\,R)$. For a splitting ideal such a presentation was firstly obtained by Swan~\cite{Swa1,Swa2,Swa3}, and for an arbitrary $I$ by Keune~\cite{Keu} and Loday~\cite{Lod}.

For a group $G$ acting on a group $H$ from the left denote by $\!\,^gh$ the image of $h\in H$ under homomorphism corresponding by $g\in G$.

Let $I\trianglelefteq R$ be a splitting ideal then $\St(n,\,R,\,I)$ can be presented as a group with action of $\St(n,\,R)$ with the set of (relative) generators
$$
\{y_{ij}(a)\mid a\in I,\ 1\leq i\neq j\leq n\}
$$
subject to relations
\setcounter{equation}{0}
\renewcommand{\theequation}{KL\arabic{equation}}
\begin{align}
&y_{ij}(a)y_{ij}(b)=y_{ij}(a+b),\\
&\!\,^{x_{ij}(r)}y_{hk}(b)=y_{hk}(b),\text{ for }h\neq j,\ k\neq i,\\
&\!\,^{x_{ij}(r)}y_{jk}(b)\cdot y_{jk}(-b)=y_{ik}(rb),\\
&y_{ij}(a)\cdot\,^{x_{jk}(s)}y_{ij}(-a)=y_{ik}(as),\\
&\!\,^{x_{hk}(a)}\big(\!\,^gy_{ij}(b)\big)=y_{hk}(a)\cdot\,^gy_{ij}(b)\cdot y_{hk}(-a)\,\text{ for }a\in I,\ g\in\St(n,\,R).
\end{align}

In other words, $\St(n,\,R,\,I)$ is isomorphic to a quotient group of a free group $F$ on the set of generators 
$$\St(n,\,R)\times\{y_{ij}(a)\mid a\in I,\ 1\leq i\neq j\leq n\}$$
(the natural action of $\St(n,\,R)$ on $F$ is given by $\!\,^f(g,\,y)=(fg,\,y)$) modulo the normal equivariant subgroup generated by KL1--KL5 (where $y_{ij}(a)$ stands for $(1,\,y_{ij}(a))$. Keune in Loday consider only stable Steinberg group $\St(R)=\varinjlim\St(n,\,R)$. This presentation for unstable group firstly appeared in~\cite{SCh} in the context of Chevalley groups. Observe that the definition of the relative Steinberg group in~\cite[Def.~3.3]{SCh} differs from ours but remark after this definition and~\cite[Lem.~8]{SCh} immediantly imply that for a splitting ideal $I\trianglelefteq R$ it coincides with the one given in the present paper. The existance of the presentation is proven in~\cite[Prop.~6]{SCh}. Relation~5 in~\cite[Prop.~6]{SCh} joins relations KL3 and KL4, and relations~2 and 3 follow from KL4 and KL5 with the use of KL6.

One could probably wonder why Tulenbaev needs his own presentation for $\St(n,\,R,\,I)$ and does not use the one of Keune--Loday. The reason is the following. To construct a map from Keune--Loday group one should firstly define an action of the absolute Steinberg group on the target group, in our context, the action of $\St(n,\,B_a)$ on $\St(n,\,B,\,I)$, what is probably possible, but seems to be much harder, then giving another presentation (in particular, one should define $\!\,^{x_{ij}(r/a^m)}y_{ji}(c)$).

Now, to construct
$
\kappa\colon\St(n,\,R,\,I)\rightarrow\St^*(n,\,R,\,I)
$
we need to define an action of $\St(n,\,R)$ on $\St^*(n,\,R,\,I)$ and find elements $y^*_{ij}(a)\in\St^*(n,\,R,\,I)$ subject to KL1--KL5. To define a action of the absolute group we use van der Kallen's another presentaion for it. For $u\in\Um(n,\,R)$, $v\in R^n$, $u^tv=0$ define
$
\alpha(u,\,v)\colon\St^*(n,\,R,\,I)\rightarrow\St^*(n,\,R,\,I)
$
by $\alpha(u,\,v)\big(F(u',\,v')\big)=F(t(u,\,v)u',\,t(u,\,v)^*v')$, and $\alpha(u,\,v)\big(S(u',\,v')\big)=S(t(u,\,v)u',\,t(u,\,v)^*v')$. Obviously, the images of the generators satisfy R1--R4, so that $\alpha(u,\,v)$ is a well-defined automorphism. Also, $\alpha$'s themselves clearly satisfy K1--K2, thus $X(u,\,v)\mapsto\alpha(u,\,v)$ is a well-defined action of $\St(n,\,R)$ on $\St^*(n,\,R,\,I)$.

Next, define $y_{ij}^*(a)=F(e_i,\,e_ja)$, $a\in I$. These elements obviously satisfy KL1, KL2 and KL5. Check KL3:
\begin{multline*}
\!\,^{x_{ij}(r)}y^*_{jk}(b)=F(t_{ij}(r)e_j,\,t_{ji}(-r)e_kb)=F(e_ir+e_j,\,e_kb)=\\
=S(e_irb+e_jb,\,e_k)=S(e_irb,\,e_k)S(e_jb,\,e_k)=F(e_i,\,e_krb)F(e_j,\,e_kb).
\end{multline*} 
KL4 is similar. Finally, we have a well-defined map
$$
\kappa\colon\St(n,\,R,\,I)\rightarrow\St^*(n,\,R,\,I)
$$
sending $y_{ij}(a)$ to $y^*_{ij}(a)$. 
\end{comment}

\section{The local-global principle for \texorpdfstring{$\Kt$}{K2}}\label{sec:lgp}
%At this point we have demonstrated for $n\geq4$ and a splitting ideal $I$ that $\St^*(n,\,R,\,I)$ and $\St(n,\,R,\,I)$ are isomorphic.
Now we turn to the main result of this paper, namely, we construct for $n\geq 4$ the map $$T\colon\St(n,\,R_a[X],\,XR_a[X])\rightarrow\St(n,\,R\ltimes XR_a[X]),$$ that fits into the diagram \eqref{eq:diagram}.

We will prove a somewhat more general result. Let $B$ be a commutative unital ring.
We call an ideal $I$ of $B$ \emph{uniquely $r$-divisible} if for every $m\in I$ there exists only one $m'\in I$ such that $rm' = m$ (we denote such $m'$ by $\frac{m}{r}$).
Clearly, $I$ is uniquely $r$-divisible if and only if the restriction of the morphism of principal localisation $\lambda_r\colon R \to R_r$ to $I$ is an isomorphism.

\begin{tm} \label{a3map} Let $B$ be a ring, $a\in B$ and let $I$ be an ideal of $B$ that is uniquely $a$-divisible.
Then for $n\geq 4$ there exists a map $T\colon\St^*(n,\,B_a,\,I)\rightarrow\St(n,\,B)$ which makes the following diagram commute.
\begin{equation} \xymatrix{ \St^*(n,\,B,\,I)\ar@<-0.0ex>[rr]^{\iota}\ar@<-0.0ex>[d]_{\lambda_{a}^*} && \St(n,\,B)\ar@<-0.0ex>[d]^{\lambda_{a}^*}\\
                            \St^*(n,\,B_a,\,I)\ar@<-0.0ex>[rr]_{\iota}\ar@{-->}[rru]_{T}            && \St(n,\,B_a)} \label{eq:a3diag} \end{equation} \end{tm}

For a vector $u\in R^n$ we denote by $I(u)$ the ideal generated by the entries of $u\in R^n$, i.\,e. $I(u)=\sum\limits_{k=1}^nu_kR$.
To prove \cref{a3map} we follow the approach of Tulenbaev and construct yet another family of elements $X_{u,v}(a)\in \St(n,\,B)$, $a\in B$ which contains van der Kallen's generators $X(u, v)$ as a subfamily (i.\,e. $X_{u,v}(1) = X(u, v)$).
The key feature of this definition is that the assumption $u \in\Um(n, B)$ is replaced with a weaker condition $a \in I(u)$ (this is equivalent to saying that $u$ becomes unimodular after localisation in $a$).
%The idea is the following. Say $a$ is not a zero divisor, then $ua^m\in B$ for some $m$ and we will construct some element ``$X(ua^m,\,va^{-m})$''.
%One can not garantee that $ua^m\in\E(n,\,B)e_1a^k$, however, the ideal generated by entries of $ua^m$ contains some power of $a$ .
%More formally, we have the following definition.
%Then, we need generators parametrised by pairs $(u,\,v)$ with $u^tv=0$, $v\in I$, $a^m\in\mathrm I(u)$ for some $m\in\mathbb N$. Such generators are defined in~\cite{Tul}.

\setcounter{df}{1}
\begin{df} \label{df:TulX}
For $u \in R^n$ we denote by $D(u)$ the set consisting of all $v\in B^n$ decomposing into a sum $v=\sum_{k=1}^Nv_k$ where $v_1,\ldots,v_N\in B^n$ are such that $u^tv_k=0$ and each $v_k$ has at least two zero coordinates.
Now for $a\in I(u)$ and $v \in D(u)$ set $X_{u,v}(a) = \prod\limits_{k=1}^Nx(u,\,v_ka)$.
\end{df}
Tulenbaev shows that the factors $x(u,\,v_ka)$, $1\leq k\leq N$ commute, see~\cite[Lemma~1.1\,e)]{Tul}.
He also shows that if $v$ admits another decomposition $v=\sum_{j=1}^Mv'_j$ satisfying the assumptions of \cref{df:TulX} then one has
$\prod_{k=1}^Nx(u,\,v_ka)=\prod_{j=1}^Mx(u,\,v'_ja)$, see the discussion following \cite[Lemma~1.1]{Tul}.
Thus, the elements $X_{u,v}(a)$ are well-defined. Obviously, $\phi(X_{u,v}(a))=t(u,\,va)$.

\begin{rk} \label{rk:ID}
From the canonical decomposition~\eqref{eq:canonical} it follows that $vb\in D(u)$ for any $b\in I(u)$ and $v$ such that $v^tu=0$.
In particular, if $I$ is uniquely $a$-divisible and $a^k\in I(u)$, then every $v\in I^n$ satisfying $v^tu=0$ is contained in $D(u)$.
\end{rk}

Notice that Tulenbaev uses different notation for van der Kallen elements, e.\,g. he writes $X_{u,v}$ instead of $X(u,\,v)$ and $X(u,\,v)$ instead of $x(u,\,v)$. 
We stick to van der Kallen's notation.

\begin{lm} \label{xproperties}
For $u$, $w\in B^n$, $v, v' \in D(u)$ such that $u^tw=0$, for $a$, $b\in I(u)$, $c\in B$, $g\in\St(n,\,B)$ the following are true:
\begin{enumerate}
\item $X_{u,vc}(a)=X_{u,v}(ca)$,
\item $X_{uc,v}(ca)=X_{u,vc^2}(a)$,
\item $X_{u,v}(a)X_{u,v'}(a)=X_{u,v+v'}(a)$,
\item $g\,X_{u,wb}(a)g\inv=X_{\phi(g)u,\phi(g)^*wb}(a)$.
\end{enumerate}
\end{lm}

\begin{proof}
The statement of $(\mathrm a)$ is obvious from the definition, $(\mathrm b)$ follows from~\cite[Lemma~1.1\,d)]{Tul}, $(\mathrm c)$ is exactly the statement of~\cite[Lemma~1.3\,a)]{Tul}.

The assertion of $(\mathrm d)$ is proven in~\cite[Lemma~1.3\,b)]{Tul} under the assumption $n\geq5$. 
Afterwards Tulenbaev remarks that the assertion remains true for $n=4$. % We will give a detailed proof of this fact.
Indeed, take $z\in B^n$ such that $z^tu=b$ and write the canonical decomposition \eqref{eq:canonical}:
$$(z^tu)w=\sum_{i<j}u_{ij} c_{ij}, \text{ where } u_{ij}=e_iu_j-e_ju_i,\text{ and }c_{ij}=w_iz_j-w_jz_i.$$
Each $u_{ij} c_{ij}$ is orthogonal to $u$ and, because of the assumption $n\geq4$, has at least two zero coordinates.
Thus, $X_{u,wb}(a)=\prod_{i<j}x(u,\, u_{ij} ac_{ij})$. 
It is enough to prove the assertion of $(\mathrm d)$ in the special case $g=x_{hk}(r)$.
If $h\neq i,j$ or $\{h,k\}=\{i,j\}$ the vector $\phi(g)^*u_{ij}ac_{ij}$ still has two zero coordinates.

Now consider the case $j=h$, $i\neq k$
By~\cite[3.12]{vdK} one has $$g\,x(u,\,u_{ij}ac_{ij})g\inv=x(\phi(g)u,\,\phi(g)^*u_{ij}ac_{ij}).$$
Set $u' = \phi(g)u$, since
$\phi(g)^*u_{ij} = t_{kj}(-r) \cdot u_{ij} = e_iu_j - e_ju_i + e_kru_i =u'_{ij}+u'_{ki}r,$ 
we obtain from~\cite[3.11]{vdK} that
$$x(\phi(g)u,\,\phi(g)^* u_{ij} ac_{ij})=x\left(u',\,u'_{ij}c_{ij}a\right)\cdot x\left(u',\,u'_{ki} rc_{ij}a\right).$$
Decomposing in this fashion each factor $g\,x(u,u_{ij}ac_{ij})g^{-1}$ for which $\phi(g)^*u_{ij}ac_{ij}$ does not have two zero coordinates, 
 we arrive at a product satisfying the requirements of the definition of $X_{\phi(g)u,\phi(g)^*wb}(a)$.

 Since $u_{ij} = - u_{ji}$, $c_{ij}= - c_{ji}$ the case $i=h$, $j\neq k$ formally follows from the one just considered. \end{proof}

We will also need the ``transposed'' analogue of Tulenbaev's elements $X_{u,v}(a)$.
\begin{df} \label{df:TulY}
Let $v\in B^n$, $a\in I(v)$, and $u \in D(v)$, i.\,e. $u = \sum\limits_{i=1}^M u_i$ for some $u_k$ having two zero components such that $v^tu_k=0$.
Set $Y_{u,v}(a)=\prod_{k=1}^Mx(u_ka,\,v)$. 
Similarly to \cref{df:TulX} this definition does not depend neither on the order of factors nor on the choice of decomposition for $u$.
\end{df}

One can repeat van der Kallen's and Tulenbaev's arguments and prove the following transposed version of Lemma~\ref{xproperties}.
We leave the proof of this lemma to the reader.
\begin{lm}
\label{yproperties}
For $u$, $u'$, $w$ and $v\in B^n$, such that $u$ and $u'$ have decomposition as in above definition, $w^tv=0$, $a$, $b\in I(v)$, $c\in B$, $g\in\St(n,\,B)$ holds
\begin{enumerate}
\item $Y_{uc,v}(a)=Y_{u,v}(ca)$,
\item $Y_{u,vc}(ca)=Y_{uc^2,v}(a)$,
\item $Y_{u,v}(a)X_{u',v}(a)=Y_{u+u',v}(a)$,
\item $g\,Y_{wb,v}(a)g\inv=Y_{\phi(g)wb,\phi(g)^*v}(a)$.
\end{enumerate}
\end{lm}

Finally, it remains to show that for a ``good'' pair $(u,\,v)$ the elements $X$ and $Y$ coincide.
\begin{lm}\label{x=y}
Let $u$, $v$, $x$, $y$ be elements of $B^n$ and $b,r\in B$ be such that $u^t v = 0$, $x^ty=b$, $x^tv=0$, $u^ty=0$ and $b\in I(u)\cap I(v)$.
Then one has $X_{u,vb^4r}(b)=Y_{ub^4r,v}(b).$
\end{lm}
\begin{proof}
Compute $g=[Y_{-xbr,v}(b),\,X_{u,yb}(b)]$ in two different ways using Lemmas~\ref{xproperties}~and~\ref{yproperties}:
$$ g = X_{t(xb^2r,-v)u,\,t(xb^2r,-v)^*yb}(b)X_{u,-yb}(b) = X_{u,\,yb+vb^4r}(b)\, X_{u,-yb}(b) = X_{u,vb^4r}(b),$$
$$ g = Y_{-xbr,v}(b) Y_{t(u,yb^2)xbr,\,t(u,yb^2)^*v}(b)= Y_{-xbr,v}(b)\, Y_{xbr+ub^4r,\,v}(b)=Y_{ub^4r,v}(b).\qedhere$$
\end{proof}

Now, we are ready to construct the desired map $T\colon\St(n,\,B_a,\,I)\rightarrow\St(n,\,B).$

\begin{proof}[Proof of Theorem~\ref{a3map}]
Consider $u=Me_1$, $M\in\E(n,\,B_a)$ and $v\in I^n$ such that $u^tv=0$.
Set $w=M^*e_1$. Since $w^tu=1$ there exist vectors $\tilde w$, $\tilde u\in B^n$ and a natural number $m$ such that 
$$ \lambda_a(\tilde{w})=wa^m,\ \lambda_a(\tilde{u})=ua^m\text{ and moreover }\tilde u^tv=0\text{ and }\tilde w^t\tilde u=a^{2m}.$$
It is clear that $a^{2m}\in I(\tilde u)$, moreover by~\cref{rk:ID} we have $v/a^{3m} \in D(\tilde u)$, therefore we are allowed to set $T(F(u,\,v))=X_{\tilde u,v/a^{3m}}(a^{2m})$. 

The first two assertions of \cref{xproperties} guarantee that this definition does not depend on the choice of $m$ and the liftings $\tilde u$ and $\tilde w$.
Similarly, we can define $T(S(u,\,v))=Y_{u/a^{3m},\tilde v}(a^{2m})$.
In view of Lemmas~\ref{xproperties}~and~\ref{yproperties} the map $T$ preserves relations \eqref{add4}--\eqref{conj3}.

It remains to check for $u = Me_1$, $v=M^*e_2$ and $c\in I$ that 
\begin{equation}T(F(u, cv)) = X_{\tilde u,\,\tilde v c/a^{4m}}(a^{2m}) = Y_{\tilde uc/a^{4m},\, \tilde v}(a^{2m}) = T(S(uc, v)). \label{eq:last}\end{equation}
Here $\tilde u, \tilde v \in B^n$ are liftings of $u$ and $v$ such that $\lambda_a(\tilde u)= ua^m$, $\lambda_a(\tilde v)= va^m$ and $\tilde u^t \tilde v = 0$.
We can also assume that $m$ is so large that $a^{2m}\in I(\tilde{u})\cap I(\tilde{v})$ and there exist $\tilde{x}$, $\tilde{y}$ such that the following equations hold:
$$\lambda_a(\tilde{x})=Me_3a^m,\ \lambda_a(\tilde{y})=M^*e_3a^m\text{ and }\tilde{x}^t\tilde{y}=a^{2m},\ \tilde{x}^t\tilde{v} = 0,\ \tilde{u}^t\tilde{y} =0.$$
To obtain~\eqref{eq:last} it remains to apply \cref{x=y} taking $b=a^{2m}$, $r=c/a^{12m}$.
Thus, the map $T$ is well-defined.

Commutativity of the diagram~\eqref{eq:a3diag} follows directly from the definitions of elements $X(u,\,v)$, $Y(u,\,v)$ and $X_{u,v}(a)$, $Y_{u,v}(a)$. \end{proof}

\printbibliography

\end{document}


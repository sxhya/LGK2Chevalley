\documentclass[12pt]{amsart}
\usepackage{amscd, amsmath, amssymb, amsthm, amsfonts, amstext, verbatim, enumitem, graphicx, mathtools, xfrac, tikz-cd, nameref, thmtools, hyperref}
\usepackage[utf8]{inputenc}
\usepackage[backend=biber, citestyle=numeric-comp, natbib=true, sortlocale=en_US, url=false, doi=false, eprint=true, maxbibnames=4]{biblatex}            
\addbibresource{paper.bib}
\usepackage[activate={true,nocompatibility}]{microtype}
\usepackage[capitalize]{cleveref}
            
\newlist{thmlist}{enumerate}{1} \setlist[thmlist]{label=(\roman{thmlisti}), ref=\thethm.(\roman{thmlisti}),noitemsep} \Crefname{thmlisti}{Theorem}{Theorems}
\newlist{lemlist}{enumerate}{1} \setlist[lemlist]{label=(\roman{lemlisti}), ref=\thelemma.(\roman{lemlisti}),noitemsep} \Crefname{lemlisti}{Lemma}{Lemmas}

\theoremstyle{plain} \declaretheorem[name=Theorem, Refname={Theorem,Theorems}]{thm} \Crefname{thm}{Theorem}{Theorems}
\numberwithin{equation}{section}

\newtheorem{lemma}{Lemma} \numberwithin{lemma}{section} \Crefname{lemma}{Lemma}{Lemmas}
\newtheorem{cor}[lemma]{Corollary} \Crefname{cor}{Corollary}{Corollaries}
\newtheorem{prop}[lemma]{Proposition} \Crefname{prop}{Proposition}{Propositions}
\newtheorem*{thm*}{Theorem}
\newtheorem*{lemma*}{Lemma}
\theoremstyle{definition} \newtheorem{dfn}[lemma]{Definition} \Crefname{dfn}{Definition}{Definitions}
\newtheorem{example}[lemma]{Example} \Crefname{example}{Example}{Examples}
\theoremstyle{remark} \newtheorem{rem}[lemma]{Remark} \Crefname{rem}{Remark}{Remarks}

%Page layout
\textwidth 16cm 
\textheight 22cm 
\headheight 0.5cm 
\evensidemargin 0.3cm 
\oddsidemargin 0.2cm

\newcommand{\catname}[1]{{\normalfont\textbf{#1}}}
\newcommand{\Ker}{\operatorname{Ker}}
\newcommand{\Cent}{\operatorname{Cent}}
\newcommand{\Img}{\operatorname{Im}}
\newcommand{\Coker}{\operatorname{Coker}}
\newcommand{\coker}{\operatorname{coker}}
\newcommand{\K}{\operatorname{\mathrm{K}}}
\newcommand{\GG}{\operatorname{\mathrm{G}}}
\newcommand{\St}{\operatorname{\mathrm{St}}}
\newcommand{\StU}{\operatorname{\hat{\mathrm{U}}}}
\newcommand{\G}{\operatorname{\mathrm{G}}}
\newcommand{\E}{\operatorname{\mathrm{E}}}
\newcommand{\M}{\operatorname{\mathrm{M}}}
\newcommand{\sr}{\operatorname{\mathrm{sr}}}
\newcommand{\Um}{\operatorname{\mathrm{Um}}}
\newcommand{\rk}{\operatorname{\mathrm{rk}}}
\newcommand{\rA}{\mathsf{A}}
\newcommand{\rB}{\mathsf{B}}
\newcommand{\rC}{\mathsf{C}} 
\newcommand{\rD}{\mathsf{D}} 
\newcommand{\rE}{\mathsf{E}}
\newcommand{\HH}{\mathrm{H}}
\newcommand{\eval}[4]{ev \textstyle \left[\frac{#2[#1] \rightarrow #3}{#1 \mapsto #4}\right]}
\newcommand{\ev}[3]{\eval{t}{#1}{#2}{#3}}

\newcommand\restr[2]{{\left.\kern-\nulldelimiterspace #1 \vphantom{\big|}\right|_{#2}}}
\newcommand*\MapsTo{\xrightarrow[\raisebox{0.25 em}{\smash{\ensuremath{\sim}}}]{}}  

\title{Local-global principle for $\K_2$ of Chevalley groups}
\keywords {Chevalley groups, Steinberg groups, $K_2$-functor. {\em Mathematical Subject Classification (2010):} 19C09; 20G35}
\author {Andrei Lavrenov, Sergey Sinchuk}
\email {avlavrenov {\it at} gmail.com, sinchukss {\it at} gmail.com}
\date {\today}

\begin{document}

\begin{abstract} To be written. \end{abstract}

\maketitle

\section {Introduction}\label{intro}
 
\subsection{Acknowledgements}
The authors of the present paper acknowledges financial support from Russian Science Foundation grant 14-11-00297.

\section {Preliminaries}\label{prelim}

\section {Main results}

\subsection{Relative Steinberg group in rank $3$ and Tulenbaev's section map} 
The main goal of this subsection is to construct Tulenbaev's section map from~\cite[\S~2]{T} for
the linear Steinberg group of rank $3$. Combining this result with the amalgamation theorem
of the the second named author (see~\cite[Theorem~9]{S15}) we deduce centrality for the orthogonal
$\K_2$ in the next subsection.

First of all, we will need to give yet another definition of the relative linear Steinberg group 
(cf. \cite[Definitions~3.3 and 3.7]{S15}).

\begin{dfn}
 The relative Steinberg group $\St^*(n,R, I)$ is the group defined by the following two
 families generators and four families of relations.
 \begin{itemize}
  \item Generators:
  \begin{enumerate}
  \item $X^1(u, v)$, where $u \in \E(n,R) \cdot e_1$, $v\ \in I^n$ such that $v^t \cdot u = 0$;
  \item $X^2(u, v)$, where $u \in I^n$, $v \in \E(n,R) \cdot e_1$ such that $v^t \cdot u = 0$.
 \end{enumerate}
  Notice that $\phi$ maps both $X^1(u, v)$ and $X^2(u, v)$ to $T(u, v) = e_n + u \cdot v^T$.
  \item Relations:
  \begin{enumerate}
  \item $X^1(u, v) \cdot X^1(u, w) = X^1(u, v+w)$, $u \in \E(n,R) \cdot e_1$, $v, w \in I^n$;
  \item $X^2(u, v) \cdot X^2(w, v) = X^2(u+w, v)$, $u, w \in I^n$, $v \in \E(n,R) \cdot e_1$;
  \item ${}^{X^\sigma(u^2, v^2)} \! X^\tau(u^1, v^1) = X^\tau(T(u^2, v^2) \cdot u^1, T(v^2, u^2)^{-1} \cdot v^1)$, $\sigma, \tau = 1..2$;
  \item $X^1(g \cdot e_1, g^* \cdot be_2) = X^2(g \cdot be_1, g^* \cdot e_2)$ where $b\in I$ and $g^* = {g^t}^{-1}$ denotes the contragradient matrix.
 \end{enumerate}
 \end{itemize}
\end{dfn}
It is easy to check the correctness of the above definition.

\begin{lemma}
 The groups $\St^*(n, R, I)$ and $\St(n, R, I)$ are isomorphic.
\end{lemma}
\begin{proof}
 TODO: 
\end{proof}

The next step of the proof is to is construct certain elements in $\St(n, R)$ similar to Tulenbaev's elements $X_{u,v}(a)$ see~\cite[\S~1]{T}.

Let $v\in R^n$ be a column.
A column $w\in R^n$ is called \emph{$v$-decomposable} if it can be presented as a finite sum $w = \sum\limits_{i=1}^n w^i$ such that each $w^i$ has at least two zero entries. 
Denote by $D(v)$ the submodule of $R^n$ consisting of all $v$-decomposable columns.
For a column $v\in R^n$ denote by $I(v)$ the ideal of $R$ spanned by coordinates $v_1,\ldots, v_n$.

Let $u,v,w\in R^n$ be columns such that $w^tv=0$.
It is easy to check (cf.~\cite[Lemma~3.2]{Ka}) that 
$$(uv)\cdot w = \sum_{i<j}w_{ij},\ \text{where}\ w_{ij} = (w_iu_j - w_ju_i)(v_je_i - v_ie_j)\in{}\!A^n.$$
The above decomposition is called the \emph{canonical} decomposition of $(uv)\cdot w$.
In particular, the above argument shows that row $a\cdot w$ is always $v$-decomposable for $a\in I(v)$, i.e. $I(v) \cdot O(v) \subseteq D(v)$.
It is also obvious that $O(v)\subseteq O(bv)$, $D(v)\subseteq D(bv)$ for $b \in R$.
%\end{enumerate}\end{lemma}

Denote by $B^1$ the subset of $R^n \times R^n \times R$ consisting of triples $(u, v, a)$ such that $v^t \cdot u = 0$, $v \in D(u)$, $a \in I(u)$.
Denote by $B^2$ the set consisting of triples $(v, u, a)$ such that $(u, v, a) \in B^1$.

\begin{lemma} \label{lem:Zproperties}
Assume that $n \geq 4$.
One can define two families of elements $Z^\tau(u, v, a) \in \St(n, R)$, $\tau=1,2$ parametrized by $(u, v, a) \in B^\tau$ satisfying the following properties
 \begin{lemlist}
  \item \label{rel:proj} $\phi(Z^\tau(u, v, a)) = e + uav^t \in \E(n, R)$, $(u,v,a) \in B^\tau$;
  \item \label{rel:add1} $Z^{1}(u, v_1 + v_2, a) = Z^{1}(u, v_1, a) \cdot Z^{1}(u, v_2, a)$;
  \item \label{rel:add2} $Z^{2}(v_1 + v_2, u, a) = Z^{2}(v_1, u, a) \cdot Z^{2}(v_2, u, a)$;
  \item \label{rel:conj} ${}^{g}\! Z^{\tau}(u, v, a) = Z^{\tau}(\phi(g) \cdot u, \phi(g)^* \cdot v, a)$, $\tau = 1,2$, $g \in \St(n, R)$.
 \end{lemlist}
\end{lemma}
\begin{proof}
One constructs the elements $Z^1(u,v,a)$ in exactly the same way as Tulenbaev constructs his elements $X_{u,v}(a)$ (see definitions preceding~\cite[Lemma~1.2]{T}).
Next, the elements $Z^2(u,v,a)$ are constructed by ``transposing'' Tulenbaev's construction.

All the properties (with the exception of the last one in the case $n=4$) can be proved in exactly the same fashion as in~\cite[Lemma~1.3]{T}.
To check the last property in the case $n=4$ one has to use the argument of W. van~der~Kallen (cf.~\cite{Ka}).
%TODO: Write more details
\end{proof}

For the rest of this section $a$ denotes a nonnilpotent element of $R$.
\begin{lemma}
Let $u,v,w,z \in R^n$ be columns such that 
\begin{equation*} u^t \cdot v = 0,\ w^t \cdot v = c \in R,\ u^t \cdot z = 0,\ w^t \cdot z = 0. \end{equation*}
Assume, moreover, that $(u, v, a) \in B^1$, $(w, z, a) \in B^2$, then one has $$Z^1(u, ac \cdot z, a) = Z^2(ac \cdot u, z, a).$$
\end{lemma}
\begin{proof}
Direct computation using \cref{lem:Zproperties} shows that
 \begin{multline*}
 Z^2(ac \cdot u, z, a) = Z^2((e+uav^t) w, (e-vau^t)z,a) \cdot Z^2(-w,z,a) = \\
  {}^{Z^1(u,v,a)} \! Z^2(w,z,a) \cdot Z^2(-w,z,a) = [Z^1(u, v, a), Z^2(w, z, a)] = Z^1(u,v,a) \cdot {}^{Z^2(w,z,a)}\! Z^1(u,-v,a) = \\
    = Z^1(u,v,a) \cdot Z^1((e+waz^t)u,-(e-zaw^t)v,a) = Z^1(u,ac \cdot z,a). \qedhere
 \end{multline*}  
\end{proof}

\subsection{Centrality of orthogonal $K_2$.} 

\printbibliography

\end{document}
